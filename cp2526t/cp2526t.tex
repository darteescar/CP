\documentclass[11pt, a4paper, fleqn]{article}
\usepackage{cp2526t}
\makeindex

%================= lhs2tex=====================================================%
%% ODER: format ==         = "\mathrel{==}"
%% ODER: format /=         = "\neq "
%
%
\makeatletter
\@ifundefined{lhs2tex.lhs2tex.sty.read}%
  {\@namedef{lhs2tex.lhs2tex.sty.read}{}%
   \newcommand\SkipToFmtEnd{}%
   \newcommand\EndFmtInput{}%
   \long\def\SkipToFmtEnd#1\EndFmtInput{}%
  }\SkipToFmtEnd

\newcommand\ReadOnlyOnce[1]{\@ifundefined{#1}{\@namedef{#1}{}}\SkipToFmtEnd}
\usepackage{amstext}
\usepackage{amssymb}
\usepackage{stmaryrd}
\DeclareFontFamily{OT1}{cmtex}{}
\DeclareFontShape{OT1}{cmtex}{m}{n}
  {<5><6><7><8>cmtex8
   <9>cmtex9
   <10><10.95><12><14.4><17.28><20.74><24.88>cmtex10}{}
\DeclareFontShape{OT1}{cmtex}{m}{it}
  {<-> ssub * cmtt/m/it}{}
\newcommand{\texfamily}{\fontfamily{cmtex}\selectfont}
\DeclareFontShape{OT1}{cmtt}{bx}{n}
  {<5><6><7><8>cmtt8
   <9>cmbtt9
   <10><10.95><12><14.4><17.28><20.74><24.88>cmbtt10}{}
\DeclareFontShape{OT1}{cmtex}{bx}{n}
  {<-> ssub * cmtt/bx/n}{}
\newcommand{\tex}[1]{\text{\texfamily#1}}	% NEU

\newcommand{\Sp}{\hskip.33334em\relax}


\newcommand{\Conid}[1]{\mathit{#1}}
\newcommand{\Varid}[1]{\mathit{#1}}
\newcommand{\anonymous}{\kern0.06em \vbox{\hrule\@width.5em}}
\newcommand{\plus}{\mathbin{+\!\!\!+}}
\newcommand{\bind}{\mathbin{>\!\!\!>\mkern-6.7mu=}}
\newcommand{\rbind}{\mathbin{=\mkern-6.7mu<\!\!\!<}}% suggested by Neil Mitchell
\newcommand{\sequ}{\mathbin{>\!\!\!>}}
\renewcommand{\leq}{\leqslant}
\renewcommand{\geq}{\geqslant}
\usepackage{polytable}

%mathindent has to be defined
\@ifundefined{mathindent}%
  {\newdimen\mathindent\mathindent\leftmargini}%
  {}%

\def\resethooks{%
  \global\let\SaveRestoreHook\empty
  \global\let\ColumnHook\empty}
\newcommand*{\savecolumns}[1][default]%
  {\g@addto@macro\SaveRestoreHook{\savecolumns[#1]}}
\newcommand*{\restorecolumns}[1][default]%
  {\g@addto@macro\SaveRestoreHook{\restorecolumns[#1]}}
\newcommand*{\aligncolumn}[2]%
  {\g@addto@macro\ColumnHook{\column{#1}{#2}}}

\resethooks

\newcommand{\onelinecommentchars}{\quad-{}- }
\newcommand{\commentbeginchars}{\enskip\{-}
\newcommand{\commentendchars}{-\}\enskip}

\newcommand{\visiblecomments}{%
  \let\onelinecomment=\onelinecommentchars
  \let\commentbegin=\commentbeginchars
  \let\commentend=\commentendchars}

\newcommand{\invisiblecomments}{%
  \let\onelinecomment=\empty
  \let\commentbegin=\empty
  \let\commentend=\empty}

\visiblecomments

\newlength{\blanklineskip}
\setlength{\blanklineskip}{0.66084ex}

\newcommand{\hsindent}[1]{\quad}% default is fixed indentation
\let\hspre\empty
\let\hspost\empty
\newcommand{\NB}{\textbf{NB}}
\newcommand{\Todo}[1]{$\langle$\textbf{To do:}~#1$\rangle$}

\EndFmtInput
\makeatother
%
%
%
%
%
%
% This package provides two environments suitable to take the place
% of hscode, called "plainhscode" and "arrayhscode". 
%
% The plain environment surrounds each code block by vertical space,
% and it uses \abovedisplayskip and \belowdisplayskip to get spacing
% similar to formulas. Note that if these dimensions are changed,
% the spacing around displayed math formulas changes as well.
% All code is indented using \leftskip.
%
% Changed 19.08.2004 to reflect changes in colorcode. Should work with
% CodeGroup.sty.
%
\ReadOnlyOnce{polycode.fmt}%
\makeatletter

\newcommand{\hsnewpar}[1]%
  {{\parskip=0pt\parindent=0pt\par\vskip #1\noindent}}

% can be used, for instance, to redefine the code size, by setting the
% command to \small or something alike
\newcommand{\hscodestyle}{}

% The command \sethscode can be used to switch the code formatting
% behaviour by mapping the hscode environment in the subst directive
% to a new LaTeX environment.

\newcommand{\sethscode}[1]%
  {\expandafter\let\expandafter\hscode\csname #1\endcsname
   \expandafter\let\expandafter\endhscode\csname end#1\endcsname}

% "compatibility" mode restores the non-polycode.fmt layout.

\newenvironment{compathscode}%
  {\par\noindent
   \advance\leftskip\mathindent
   \hscodestyle
   \let\\=\@normalcr
   \let\hspre\(\let\hspost\)%
   \pboxed}%
  {\endpboxed\)%
   \par\noindent
   \ignorespacesafterend}

\newcommand{\compaths}{\sethscode{compathscode}}

% "plain" mode is the proposed default.
% It should now work with \centering.
% This required some changes. The old version
% is still available for reference as oldplainhscode.

\newenvironment{plainhscode}%
  {\hsnewpar\abovedisplayskip
   \advance\leftskip\mathindent
   \hscodestyle
   \let\hspre\(\let\hspost\)%
   \pboxed}%
  {\endpboxed%
   \hsnewpar\belowdisplayskip
   \ignorespacesafterend}

\newenvironment{oldplainhscode}%
  {\hsnewpar\abovedisplayskip
   \advance\leftskip\mathindent
   \hscodestyle
   \let\\=\@normalcr
   \(\pboxed}%
  {\endpboxed\)%
   \hsnewpar\belowdisplayskip
   \ignorespacesafterend}

% Here, we make plainhscode the default environment.

\newcommand{\plainhs}{\sethscode{plainhscode}}
\newcommand{\oldplainhs}{\sethscode{oldplainhscode}}
\plainhs

% The arrayhscode is like plain, but makes use of polytable's
% parray environment which disallows page breaks in code blocks.

\newenvironment{arrayhscode}%
  {\hsnewpar\abovedisplayskip
   \advance\leftskip\mathindent
   \hscodestyle
   \let\\=\@normalcr
   \(\parray}%
  {\endparray\)%
   \hsnewpar\belowdisplayskip
   \ignorespacesafterend}

\newcommand{\arrayhs}{\sethscode{arrayhscode}}

% The mathhscode environment also makes use of polytable's parray 
% environment. It is supposed to be used only inside math mode 
% (I used it to typeset the type rules in my thesis).

\newenvironment{mathhscode}%
  {\parray}{\endparray}

\newcommand{\mathhs}{\sethscode{mathhscode}}

% texths is similar to mathhs, but works in text mode.

\newenvironment{texthscode}%
  {\(\parray}{\endparray\)}

\newcommand{\texths}{\sethscode{texthscode}}

% The framed environment places code in a framed box.

\def\codeframewidth{\arrayrulewidth}
\RequirePackage{calc}

\newenvironment{framedhscode}%
  {\parskip=\abovedisplayskip\par\noindent
   \hscodestyle
   \arrayrulewidth=\codeframewidth
   \tabular{@{}|p{\linewidth-2\arraycolsep-2\arrayrulewidth-2pt}|@{}}%
   \hline\framedhslinecorrect\\{-1.5ex}%
   \let\endoflinesave=\\
   \let\\=\@normalcr
   \(\pboxed}%
  {\endpboxed\)%
   \framedhslinecorrect\endoflinesave{.5ex}\hline
   \endtabular
   \parskip=\belowdisplayskip\par\noindent
   \ignorespacesafterend}

\newcommand{\framedhslinecorrect}[2]%
  {#1[#2]}

\newcommand{\framedhs}{\sethscode{framedhscode}}

% The inlinehscode environment is an experimental environment
% that can be used to typeset displayed code inline.

\newenvironment{inlinehscode}%
  {\(\def\column##1##2{}%
   \let\>\undefined\let\<\undefined\let\\\undefined
   \newcommand\>[1][]{}\newcommand\<[1][]{}\newcommand\\[1][]{}%
   \def\fromto##1##2##3{##3}%
   \def\nextline{}}{\) }%

\newcommand{\inlinehs}{\sethscode{inlinehscode}}

% The joincode environment is a separate environment that
% can be used to surround and thereby connect multiple code
% blocks.

\newenvironment{joincode}%
  {\let\orighscode=\hscode
   \let\origendhscode=\endhscode
   \def\endhscode{\def\hscode{\endgroup\def\@currenvir{hscode}\\}\begingroup}
   %\let\SaveRestoreHook=\empty
   %\let\ColumnHook=\empty
   %\let\resethooks=\empty
   \orighscode\def\hscode{\endgroup\def\@currenvir{hscode}}}%
  {\origendhscode
   \global\let\hscode=\orighscode
   \global\let\endhscode=\origendhscode}%

\makeatother
\EndFmtInput
%
%%format (bin (n) (k)) = "\Big(\vcenter{\xymatrix@R=1pt{" n "\\" k "}}\Big)"
%------------------------------------------------------------------------------%


%====== DEFINIR GRUPO E ELEMENTOS =============================================%

\group{G05}
\studentA{106936}{Duarte Escairo }
\studentB{106932}{Luís Soares }
\studentC{106856}{Tiago Figueiredo }

%==============================================================================%

\begin{document}
\sffamily
\setlength{\parindent}{0em}
\emergencystretch 3em
\renewcommand{\baselinestretch}{1.25} 
\input{Cover}
\pagestyle{pagestyle}

\newgeometry{left=25mm,right=20mm,top=25mm,bottom=25mm}
\setlength{\parindent}{1em}

\section*{Preâmbulo}

Em \CP\ pretende-se ensinar a progra\-mação de computadores
como uma disciplina científica. Para isso parte-se de um repertório de \emph{combinadores}
que formam uma álgebra da programação % (conjunto de leis universais e seus corolários)
e usam-se esses combinadores para construir programas \emph{composicionalmente},
isto é, agregando programas já existentes.

Na sequência pedagógica dos planos de estudo dos cursos que têm
esta disciplina, opta-se pela aplicação deste método à programação
em \Haskell\ (sem prejuízo da sua aplicação a outras linguagens
funcionais). Assim, o presente trabalho prático coloca os
alunos perante problemas concretos que deverão ser implementados em
\Haskell. Há ainda um outro objectivo: o de ensinar a documentar
programas, a validá-los e a produzir textos técnico-científicos de
qualidade.

Antes de abordarem os problemas propostos no trabalho, os grupos devem ler
com atenção o anexo \ref{sec:documentacao} onde encontrarão as instruções
relativas ao \emph{software} a instalar, etc.

Valoriza-se a escrita de \emph{pouco} código que corresponda a soluções
simples e elegantes que utilizem os combinadores de ordem superior estudados
na disciplina.

\noindent \textbf{Avaliação}. Faz parte da avaliação do trabalho a sua defesa
por parte dos elementos de cada grupo. Estes devem estar preparados para
responder a perguntas sobre \emph{qualquer} dos problemas deste enunciado.
A prestação \emph{individual} de cada aluno nessa defesa oral será uma componente
importante e diferenciadora da avaliação.


\Problema

Uma serialização (ou travessia) de uma árvore é uma sua representação sob a forma de uma lista. 
Na biblioteca \ensuremath{\Conid{BTree}} encontram-se as funções de serialização \ensuremath{\Varid{inordt}}, \ensuremath{\Varid{preordt}} e \ensuremath{\Varid{postordt}},
que fazem as travessias \emph{in-order}, \emph{ pre-order} e \emph{post-order}, respectivamente.
Todas essas travessias são catamorfismos que percorrem a árvore argumento em regime \emph{depth-first}.

Pretende-se agora uma função \ensuremath{\Varid{bforder}} que faça a travessia em regime \emph{breadth-first},
isto é, por níveis.
Por exemplo, para a árvore \ensuremath{t_1 } dada em anexo e mostrada na figura a seguir,

\begin{center}
	\figura
\end{center}

\noindent a função deverá dar a lista

\begin{hscode}\SaveRestoreHook
\column{B}{@{}>{\hspre}l<{\hspost}@{}}%
\column{9}{@{}>{\hspre}l<{\hspost}@{}}%
\column{E}{@{}>{\hspre}l<{\hspost}@{}}%
\>[9]{}[\mskip1.5mu \mathrm{5},\mathrm{3},\mathrm{7},\mathrm{1},\mathrm{4},\mathrm{6},\mathrm{8}\mskip1.5mu]{}\<[E]%
\ColumnHook
\end{hscode}\resethooks

\noindent em que se vê como os níveis \ensuremath{\mathrm{5}}, depois \ensuremath{\mathrm{3},\mathrm{7}} e finalmente \ensuremath{\mathrm{1},\mathrm{4},\mathrm{6},\mathrm{8}} foram percorridos.

Pretendemos propor duas versões dessa função:

\begin{enumerate}
\item	Uma delas envolve um catamorfismo de \ensuremath{\Conid{BTree}}s:
\begin{hscode}\SaveRestoreHook
\column{B}{@{}>{\hspre}l<{\hspost}@{}}%
\column{E}{@{}>{\hspre}l<{\hspost}@{}}%
\>[B]{}\Varid{bfsLevels}\mathbin{::}\Conid{BTree}\;\Varid{a}\to [\mskip1.5mu \Varid{a}\mskip1.5mu]{}\<[E]%
\\
\>[B]{}\Varid{bfsLevels}\mathrel{=}\Varid{concat}\comp \Varid{levels}{}\<[E]%
\ColumnHook
\end{hscode}\resethooks
Complete a definição desse catamorfismo:
\begin{hscode}\SaveRestoreHook
\column{B}{@{}>{\hspre}l<{\hspost}@{}}%
\column{E}{@{}>{\hspre}l<{\hspost}@{}}%
\>[B]{}\Varid{levels}\mathbin{::}\Conid{BTree}\;\Varid{a}\to [\mskip1.5mu [\mskip1.5mu \Varid{a}\mskip1.5mu]\mskip1.5mu]{}\<[E]%
\\
\>[B]{}\Varid{levels}\mathrel{=}\llparenthesis\, \Varid{glevels}\,\rrparenthesis{}\<[E]%
\ColumnHook
\end{hscode}\resethooks
\item A segunda proposta,
\begin{hscode}\SaveRestoreHook
\column{B}{@{}>{\hspre}l<{\hspost}@{}}%
\column{E}{@{}>{\hspre}l<{\hspost}@{}}%
\>[B]{}\Varid{bft}\mathbin{::}\Conid{BTree}\;\Varid{a}\to [\mskip1.5mu \Varid{a}\mskip1.5mu]{}\<[E]%
\ColumnHook
\end{hscode}\resethooks
deverá basear-se num anamorfismo de listas.
\end{enumerate}
\textbf{Sugestão}: estudar o artigo \cite{Ok00} cujo PDF está incluído no material deste trabalho. 
Quando fizer testes ao seu código pode, se desejar, usar funções disponíveis na biblioteca
\ensuremath{\Conid{Exp}} para visualizar as árvores em GraphViz (formato .dot).

Justifique devidamente a sua resolução, que deverá vir acompanhada de diagramas explicativos.
Como já se disse, valoriza-se a escrita de \emph{pouco} código que corresponda a soluções
simples e elegantes que utilizem os combinadores de ordem superior estudados
na disciplina.



\Problema

Considere a seguinte função em Haskell:
\begin{quote}
\begin{hscode}\SaveRestoreHook
\column{B}{@{}>{\hspre}l<{\hspost}@{}}%
\column{10}{@{}>{\hspre}l<{\hspost}@{}}%
\column{11}{@{}>{\hspre}l<{\hspost}@{}}%
\column{23}{@{}>{\hspre}l<{\hspost}@{}}%
\column{32}{@{}>{\hspre}l<{\hspost}@{}}%
\column{39}{@{}>{\hspre}l<{\hspost}@{}}%
\column{46}{@{}>{\hspre}l<{\hspost}@{}}%
\column{52}{@{}>{\hspre}l<{\hspost}@{}}%
\column{E}{@{}>{\hspre}l<{\hspost}@{}}%
\>[B]{}\Varid{f}\;\Varid{x}\mathrel{=}\Varid{wrapper}\comp \Varid{worker}\;\mathbf{where}{}\<[E]%
\\
\>[B]{}\hsindent{10}{}\<[10]%
\>[10]{}\Varid{wrapper}\mathrel{=}\Varid{head}{}\<[E]%
\\
\>[B]{}\hsindent{10}{}\<[10]%
\>[10]{}\Varid{worker}\;\mathrm{0}\mathrel{=}\Varid{start}\;\Varid{x}{}\<[E]%
\\
\>[B]{}\hsindent{10}{}\<[10]%
\>[10]{}\Varid{worker}\;(\Varid{n}\mathbin{+}\mathrm{1})\mathrel{=}\Varid{loop}\;\Varid{x}\;(\Varid{worker}\;\Varid{n}){}\<[E]%
\\[\blanklineskip]%
\>[B]{}\Varid{loop}\;\Varid{x}\;{}\<[11]%
\>[11]{}[\mskip1.5mu \Varid{s},{}\<[23]%
\>[23]{}\Varid{h},{}\<[32]%
\>[32]{}\Varid{k},{}\<[39]%
\>[39]{}\Varid{j},{}\<[46]%
\>[46]{}\Varid{m}{}\<[52]%
\>[52]{}\mskip1.5mu]\mathrel{=}{}\<[E]%
\\
\>[11]{}[\mskip1.5mu \Varid{h}\mathbin{/}\Varid{k}\mathbin{+}\Varid{s},\Varid{x}\mathbin{\uparrow}\mathrm{2}\mathbin{*}\Varid{h},\Varid{k}\mathbin{*}\Varid{j},\Varid{j}\mathbin{+}\Varid{m},\Varid{m}\mathbin{+}\mathrm{8}\mskip1.5mu]{}\<[E]%
\\[\blanklineskip]%
\>[B]{}\Varid{start}\;\Varid{x}\mathrel{=}[\mskip1.5mu \Varid{x},{}\<[23]%
\>[23]{}\Varid{x}\mathbin{\uparrow}\mathrm{3},{}\<[32]%
\>[32]{}\mathrm{6},{}\<[39]%
\>[39]{}\mathrm{20},{}\<[46]%
\>[46]{}\mathrm{22}{}\<[52]%
\>[52]{}\mskip1.5mu]{}\<[E]%
\ColumnHook
\end{hscode}\resethooks
\end{quote}
Pode-se provar pela lei de recursividade mútua que \ensuremath{\Varid{f}\;\Varid{x}\;\Varid{n}} calcula o seno hiperbólico de \ensuremath{\Varid{x}},
\ensuremath{\Varid{sinh}\;\Varid{x}}, para \ensuremath{\Varid{n}} aproximações da sua série de Taylor. 
Faça a derivação da função dada a partir da referida série de Taylor, apresentando todos os cálculos justificativos, tal como se faz para outras funções no capítulo respectivo do texto base desta UC \cite{Ol98-24}.

\Problema

Quem em Braga observar, ao fim da tarde, o tráfego onde a Avenida Clairmont
Fernand se junta à N101, aproximadamente na coordenada \href{https://maps.app.goo.gl/uCbXLsdibYoochr36}{41°33'46.8"N
8°24'32.4"W} --- ver as setas da figura que se segue --- reparará nas sequências
imparáveis (infinitas!) de veículos provenientes dessas vias de circulação.

Mas também irá observar um comportamento interessante por parte dos condutores desses
veículos: por regra, \emph{cada carro numa via deixa passar, à sua frente, exactamente outro carro da outra via}. 

\begin{center}
	\mapa
\end{center}

Este comportamento \emph{civilizado} chama-se \emph{fair-merge} (ou \emph{fair-interleaving})
de duas sequências infinitas, também designadas \emph{streams} em ciência
da computação. Seja dado o tipo dessas sequências em Haskell,
\begin{hscode}\SaveRestoreHook
\column{B}{@{}>{\hspre}l<{\hspost}@{}}%
\column{E}{@{}>{\hspre}l<{\hspost}@{}}%
\>[B]{}\mathbf{data}\;\Conid{Stream}\;\Varid{a}\mathrel{=}\Conid{Cons}\;(\Varid{a},\Conid{Stream}\;\Varid{a})\;\mathbf{deriving}\;\Conid{Show}{}\<[E]%
\ColumnHook
\end{hscode}\resethooks
para o qual se define também:
\begin{hscode}\SaveRestoreHook
\column{B}{@{}>{\hspre}l<{\hspost}@{}}%
\column{E}{@{}>{\hspre}l<{\hspost}@{}}%
\>[B]{}\mathsf{out}\;(\Conid{Cons}\;(\Varid{x},\Varid{xs}))\mathrel{=}(\Varid{x},\Varid{xs}){}\<[E]%
\ColumnHook
\end{hscode}\resethooks
\noindent O referido comportamento civilizado pode definir-se, em Haskell,
da forma seguinte:\footnote{O facto das sequências serem infinitas não nos
deve preocupar, pois em Haskell isso é lidado de forma transparente por \lazy{lazy
evaluation}.}
\begin{hscode}\SaveRestoreHook
\column{B}{@{}>{\hspre}l<{\hspost}@{}}%
\column{4}{@{}>{\hspre}l<{\hspost}@{}}%
\column{E}{@{}>{\hspre}l<{\hspost}@{}}%
\>[B]{}\Varid{fair\char95 merge}\mathbin{::}(\Conid{Stream}\;\Varid{a},\Conid{Stream}\;\Varid{a})+(\Conid{Stream}\;\Varid{a},\Conid{Stream}\;\Varid{a})\to \Conid{Stream}\;\Varid{a}{}\<[E]%
\\
\>[B]{}\Varid{fair\char95 merge}\mathrel{=}\alt{\Varid{h}}{\Varid{k}}\;\mathbf{where}{}\<[E]%
\\
\>[B]{}\hsindent{4}{}\<[4]%
\>[4]{}\Varid{h}\;(\Conid{Cons}\;(\Varid{x},\Varid{xs}),\Varid{y})\mathrel{=}\Conid{Cons}\;(\Varid{x},\Varid{k}\;(\Varid{xs},\Varid{y})){}\<[E]%
\\
\>[B]{}\hsindent{4}{}\<[4]%
\>[4]{}\Varid{k}\;(\Varid{x},\Conid{Cons}\;(\Varid{y},\Varid{ys}))\mathrel{=}\Conid{Cons}\;(\Varid{y},\Varid{h}\;(\Varid{x},\Varid{ys})){}\<[E]%
\ColumnHook
\end{hscode}\resethooks

Defina \ensuremath{\Varid{fair\char95 merge}} como um \textbf{anamorfismo} de \ensuremath{\Conid{Stream}}s, usando o combinador
\begin{hscode}\SaveRestoreHook
\column{B}{@{}>{\hspre}l<{\hspost}@{}}%
\column{E}{@{}>{\hspre}l<{\hspost}@{}}%
\>[B]{}\lanabracket\,\Varid{g}\,\ranabracket\mathrel{=}\Conid{Cons}\comp (\Varid{id}\times\lanabracket\,\Varid{g}\,\ranabracket)\comp \Varid{g}{}\<[E]%
\ColumnHook
\end{hscode}\resethooks
e a seguinte estratégia:
\begin{itemize}
\item	Derivar a lei \textbf{dual} da recursividade mútua,
\begin{eqnarray}
	\ensuremath{\alt{\Varid{f}}{\Varid{g}}\mathrel{=}\ana{\alt{\Varid{h}}{\Varid{k}}}} & \equiv & \ensuremath{\begin{lcbr}\Varid{out}\comp \Varid{f}\mathrel{=}\fun F \;\alt{\Varid{f}}{\Varid{g}}\comp \Varid{h}\\\Varid{out}\comp \Varid{g}\mathrel{=}\fun F \;\alt{\Varid{f}}{\Varid{g}}\comp \Varid{k}\end{lcbr}}
	\label{eq:fokkinga_dual}
\end{eqnarray}
	tal como se fez, nas aulas, para a que está no formulário.
\item
	Usar (\ref{eq:fokkinga_dual}) na resolução do problema proposto. 
\end{itemize}
Justificar devidamente a resolução, que deverá vir acompanhada de diagramas explicativos.

\Problema

Como se sabe, é possível pensarmos em catamorfismos, anamorfismos etc \emph{probabilísticos},
quer dizer, programas recursivos que dão distribuições como resultados. Por
exemplo, podemos pensar num combinador
\begin{hscode}\SaveRestoreHook
\column{B}{@{}>{\hspre}l<{\hspost}@{}}%
\column{E}{@{}>{\hspre}l<{\hspost}@{}}%
\>[B]{}\Varid{pcataList}\mathbin{::}(()+(\Varid{a},\Varid{b})\to \fun{Dist}\;\Varid{b})\to [\mskip1.5mu \Varid{a}\mskip1.5mu]\to \fun{Dist}\;\Varid{b}{}\<[E]%
\ColumnHook
\end{hscode}\resethooks
que é muito parecido com
\begin{hscode}\SaveRestoreHook
\column{B}{@{}>{\hspre}l<{\hspost}@{}}%
\column{E}{@{}>{\hspre}l<{\hspost}@{}}%
\>[B]{}\llparenthesis\, \cdot \,\rrparenthesis\mathbin{::}(()+(\Varid{a},\Varid{b})\to \Varid{b})\to [\mskip1.5mu \Varid{a}\mskip1.5mu]\to \Varid{b}{}\<[E]%
\ColumnHook
\end{hscode}\resethooks
da biblioteca \List. A principal diferença é que o gene de \ensuremath{\Varid{pcataList}} é uma função probabilística.

Como exemplo de utilização, recorde-se que \ensuremath{\llparenthesis\, \alt{\Varid{zero}}{\Varid{add}}\,\rrparenthesis} soma todos
os elementos da lista argumento, por exemplo:
\begin{quote}
\ensuremath{\llparenthesis\, \alt{\Varid{zero}}{\Varid{add}}\,\rrparenthesis\;[\mskip1.5mu \mathrm{20},\mathrm{10},\mathrm{5}\mskip1.5mu]\mathrel{=}\mathrm{35}}.
\end{quote}
Considere-se agora a função \ensuremath{\Varid{padd}} (adição probabilística) que,
com probabilidade $90\%$ soma dois números e com probabilidade $10\%$ os subtrai:
\begin{hscode}\SaveRestoreHook
\column{B}{@{}>{\hspre}l<{\hspost}@{}}%
\column{E}{@{}>{\hspre}l<{\hspost}@{}}%
\>[B]{}\Varid{padd}\;(\Varid{a},\Varid{b})\mathrel{=}\Conid{D}\;[\mskip1.5mu (\Varid{a}\mathbin{+}\Varid{b},\mathrm{0.9}),(\Varid{a}\mathbin{-}\Varid{b},\mathrm{0.1})\mskip1.5mu]{}\<[E]%
\ColumnHook
\end{hscode}\resethooks
Se se correr
\begin{hscode}\SaveRestoreHook
\column{B}{@{}>{\hspre}l<{\hspost}@{}}%
\column{E}{@{}>{\hspre}l<{\hspost}@{}}%
\>[B]{}\Varid{d4}\mathrel{=}\Varid{pcataList}\;\alt{\Varid{pzero}}{\Varid{padd}}\;[\mskip1.5mu \mathrm{20},\mathrm{10},\mathrm{5}\mskip1.5mu]\;\mathbf{where}\;\Varid{pzero}\mathrel{=}\Varid{return}\comp \Varid{zero}{}\<[E]%
\ColumnHook
\end{hscode}\resethooks
obter-se-á:
\begin{Verbatim}[fontsize=\small]
35  81.0%
25   9.0%
 5   9.0%
15   1.0%
\end{Verbatim}

Com base neste exemplo, resolva o seguinte
\begin{quote}\em
\textbf{Problema}: Uma unidade militar pretende enviar uma mensagem urgente
a outra, mas tem o aparelho de telegrafia meio avariado. Por experiência,
o telegrafista sabe que a probabilidade de uma palavra se perder (não ser
transmitida) é $5\%$; e que, no final de cada mensagem, o aparelho envia o código
\ensuremath{\text{\ttfamily \char34 stop\char34}}, mas (por estar meio avariado), falha $10\%$ das vezes.

Qual a probabilidade de a palavra \ensuremath{\text{\ttfamily \char34 atacar\char34}} da mensagem 
\begin{quote}
\ensuremath{\Varid{words}\;\text{\ttfamily \char34 Vamos~atacar~hoje\char34}}
\end{quote}
se perder, isto é, o resultado da transmissão ser \ensuremath{[\mskip1.5mu \text{\ttfamily \char34 Vamos\char34},\text{\ttfamily \char34 hoje\char34},\text{\ttfamily \char34 stop\char34}\mskip1.5mu]}?
E a de seguirem todas as palavras, mas faltar o \ensuremath{\text{\ttfamily \char34 stop\char34}} no fim? E a da transmissão
ser perfeita?
\end{quote}

Responda a estas perguntas encontrando \ensuremath{\Varid{gene}} tal que
\begin{hscode}\SaveRestoreHook
\column{B}{@{}>{\hspre}l<{\hspost}@{}}%
\column{E}{@{}>{\hspre}l<{\hspost}@{}}%
\>[B]{}\Varid{transmitir}\mathrel{=}\Varid{pcataList}\;\Varid{gene}{}\<[E]%
\ColumnHook
\end{hscode}\resethooks
descreve o comportamento do aparelho.
Justificar devidamente a resolução, que deverá vir acompanhada de diagramas explicativos.
%

\part*{Anexos}

\appendix

\section{Natureza do trabalho a realizar}
\label{sec:documentacao}
Este trabalho teórico-prático deve ser realizado por grupos de 3 alunos.
Os detalhes da avaliação (datas para submissão do relatório e sua defesa
oral) são os que forem publicados na \cp{página da disciplina} na \emph{internet}.

Recomenda-se uma abordagem participativa dos membros do grupo em \textbf{todos}
os exercícios do trabalho, para assim poderem responder a qualquer questão
colocada na \emph{defesa oral} do relatório.

Para cumprir de forma integrada os objectivos do trabalho vamos recorrer
a uma técnica de programa\-ção dita ``\litp{literária}'' \cite{Kn92}, cujo
princípio base é o seguinte:
%
\begin{quote}\em
        Um programa e a sua documentação devem coincidir.
\end{quote}
%
Por outras palavras, o \textbf{código fonte} e a \textbf{documentação} de um
programa deverão estar no mesmo ficheiro.

O ficheiro \texttt{cp2526t.pdf} que está a ler é já um exemplo de
\litp{programação literária}: foi gerado a partir do texto fonte
\texttt{cp2526t.lhs}\footnote{O sufixo `lhs' quer dizer
\emph{\lhaskell{literate Haskell}}.} que encontrará no \MaterialPedagogico\
desta disciplina des\-com\-pactando o ficheiro \texttt{cp2526t.zip}.

Como se mostra no esquema abaixo, de um único ficheiro (\ensuremath{\Varid{lhs}})
gera-se um PDF ou faz-se a interpretação do código \Haskell\ que ele inclui:

        \esquema

Vê-se assim que, para além do \GHCi, serão necessários os executáveis \PdfLatex\ e
\LhsToTeX. Para facilitar a instalação e evitar problemas de versões e
conflitos com sistemas operativos, é recomendado o uso do \Docker\ tal como
a seguir se descreve.

\section{Docker} \label{sec:docker}

Recomenda-se o uso do \container\ cuja imagem é gerada pelo \Docker\ a partir do ficheiro
\texttt{Dockerfile} que se encontra na diretoria que resulta de descompactar
\texttt{cp2526t.zip}. Este \container\ deverá ser usado na execução
do \GHCi\ e dos comandos relativos ao \Latex. (Ver também a \texttt{Makefile}
que é disponibilizada.)

Após \href{https://docs.docker.com/engine/install/}{instalar o Docker} e
descarregar o referido zip com o código fonte do trabalho,
basta executar os seguintes comandos:
\begin{Verbatim}[fontsize=\small]
    $ docker build -t cp2526t .
    $ docker run -v ${PWD}:/cp2526t -it cp2526t
\end{Verbatim}
\textbf{NB}: O objetivo é que o container\ seja usado \emph{apenas} 
para executar o \GHCi\ e os comandos relativos ao \Latex.
Deste modo, é criado um \textit{volume} (cf.\ a opção \texttt{-v \$\{PWD\}:/cp2526t}) 
que permite que a diretoria em que se encontra na sua máquina local 
e a diretoria \texttt{/cp2526t} no \container\ sejam partilhadas.

Pretende-se então que visualize/edite os ficheiros na sua máquina local e que
os compile no \container, executando:
\begin{Verbatim}[fontsize=\small]
    $ lhs2TeX cp2526t.lhs > cp2526t.tex
    $ pdflatex cp2526t
\end{Verbatim}
\LhsToTeX\ é o pre-processador que faz ``pretty printing'' de código Haskell
em \Latex\ e que faz parte já do \container. Alternativamente, basta executar
\begin{Verbatim}[fontsize=\small]
    $ make
\end{Verbatim}
para obter o mesmo efeito que acima.

Por outro lado, o mesmo ficheiro \texttt{cp2526t.lhs} é executável e contém
o ``kit'' básico, escrito em \Haskell, para realizar o trabalho. Basta executar
\begin{Verbatim}[fontsize=\small]
    $ ghci cp2526t.lhs
\end{Verbatim}

\noindent Abra o ficheiro \texttt{cp2526t.lhs} no seu editor de texto preferido
e verifique que assim é: todo o texto que se encontra dentro do ambiente
\begin{quote}\small\tt
\text{\ttfamily \char92{}begin\char123{}code\char125{}}
\\ ... \\
\text{\ttfamily \char92{}end\char123{}code\char125{}}
\end{quote}
é seleccionado pelo \GHCi\ para ser executado.

\section{Em que consiste o TP}

Em que consiste, então, o \emph{relatório} a que se referiu acima?
É a edição do texto que está a ser lido, preenchendo o anexo \ref{sec:resolucao}
com as respostas. O relatório deverá conter ainda a identificação dos membros
do grupo de trabalho, no local respectivo da folha de rosto.

Para gerar o PDF integral do relatório deve-se ainda correr os comando seguintes,
que actualizam a bibliografia (com \Bibtex) e o índice remissivo (com \Makeindex),
\begin{Verbatim}[fontsize=\small]
    $ bibtex cp2526t.aux
    $ makeindex cp2526t.idx
\end{Verbatim}
e recompilar o texto como acima se indicou. (Como já se disse, pode fazê-lo
correndo simplesmente \texttt{make} no \container.)

No anexo \ref{sec:codigo} disponibiliza-se algum código \Haskell\ relativo
aos problemas que são colocados. Esse anexo deverá ser consultado e analisado
à medida que isso for necessário.

Deve ser feito uso da \litp{programação literária} para documentar bem o código que se
desenvolver, em particular fazendo diagramas explicativos do que foi feito e
tal como se explica no anexo \ref{sec:diagramas} que se segue.

\section{Como exprimir cálculos e diagramas em LaTeX/lhs2TeX} \label{sec:diagramas}

Como primeiro exemplo, estudar o texto fonte (\lhstotex{lhs}) do que está a ler\footnote{
Procure e.g.\ por \texttt{"sec:diagramas"}.} onde se obtém o efeito seguinte:\footnote{Exemplos
tirados de \cite{Ol98-24}.}
\begin{eqnarray*}
\start
\ensuremath{\Varid{id}\mathrel{=}\conj{\Varid{f}}{\Varid{g}}}
\just\equiv{ universal property }
\ensuremath{\begin{lcbr}\p1\comp \Varid{id}\mathrel{=}\Varid{f}\\\p2\comp \Varid{id}\mathrel{=}\Varid{g}\end{lcbr}}
\just\equiv{ identity }
\ensuremath{\begin{lcbr}\p1\mathrel{=}\Varid{f}\\\p2\mathrel{=}\Varid{g}\end{lcbr}}
\qed
\end{eqnarray*}

Os diagramas podem ser produzidos recorrendo à \emph{package} \Xymatrix, por exemplo:
\begin{eqnarray*}
\xymatrix@C=2cm{
    \ensuremath{\N_0}
           \ar[d]_-{\ensuremath{\cataNat{\Varid{g}}}}
&
    \ensuremath{\mathrm{1}\mathbin{+}\N_0}
           \ar[d]^{\ensuremath{\Varid{id}\mathbin{+}\cataNat{\Varid{g}}}}
           \ar[l]_-{\ensuremath{\mathsf{in}}}
\\
     \ensuremath{\Conid{B}}
&
     \ensuremath{\mathrm{1}\mathbin{+}\Conid{B}}
           \ar[l]^-{\ensuremath{\Varid{g}}}
}
\end{eqnarray*}

\section{O mónade das distribuições probabilísticas} \label{sec:probabilities}
Mónades são functores com propriedades adicionais que nos permitem obter
efeitos especiais em progra\-mação. Por exemplo, a biblioteca \Probability\
oferece um mónade para abordar problemas de probabilidades. Nesta biblioteca,
o conceito de distribuição estatística é captado pelo tipo
\begin{eqnarray}
     \ensuremath{\mathbf{newtype}\;\fun{Dist}\;\Varid{a}\mathrel{=}\Conid{D}\;\{\mskip1.5mu \Varid{unD}\mathbin{::}[\mskip1.5mu (\Varid{a},\Conid{ProbRep})\mskip1.5mu]\mskip1.5mu\}}
     \label{eq:Dist}
\end{eqnarray}
em que \ensuremath{\Conid{ProbRep}} é um real de \ensuremath{\mathrm{0}} a \ensuremath{\mathrm{1}}, equivalente a uma escala de $0$ a
$100 \%$.

Cada par \ensuremath{(\Varid{a},\Varid{p})} numa distribuição \ensuremath{\Varid{d}\mathbin{::}\fun{Dist}\;\Varid{a}} indica que a probabilidade
de \ensuremath{\Varid{a}} é \ensuremath{\Varid{p}}, devendo ser garantida a propriedade de  que todas as probabilidades
de \ensuremath{\Varid{d}} somam $100\%$.
Por exemplo, a seguinte distribuição de classificações por escalões de $A$ a $E$,
\[
\begin{array}{ll}
A & \rule{2mm}{3pt}\ 2\%\\
B & \rule{12mm}{3pt}\ 12\%\\
C & \rule{29mm}{3pt}\ 29\%\\
D & \rule{35mm}{3pt}\ 35\%\\
E & \rule{22mm}{3pt}\ 22\%\\
\end{array}
\]
será representada pela distribuição
\begin{hscode}\SaveRestoreHook
\column{B}{@{}>{\hspre}l<{\hspost}@{}}%
\column{E}{@{}>{\hspre}l<{\hspost}@{}}%
\>[B]{}\Varid{d1}\mathbin{::}\fun{Dist}\;\Conid{Char}{}\<[E]%
\\
\>[B]{}\Varid{d1}\mathrel{=}\Conid{D}\;[\mskip1.5mu (\text{\ttfamily 'A'},\mathrm{0.02}),(\text{\ttfamily 'B'},\mathrm{0.12}),(\text{\ttfamily 'C'},\mathrm{0.29}),(\text{\ttfamily 'D'},\mathrm{0.35}),(\text{\ttfamily 'E'},\mathrm{0.22})\mskip1.5mu]{}\<[E]%
\ColumnHook
\end{hscode}\resethooks
que o \GHCi\ mostrará assim:
\begin{Verbatim}[fontsize=\small]
'D'  35.0%
'C'  29.0%
'E'  22.0%
'B'  12.0%
'A'   2.0%
\end{Verbatim}
É possível definir geradores de distribuições, por exemplo distribuições \emph{uniformes},
\begin{hscode}\SaveRestoreHook
\column{B}{@{}>{\hspre}l<{\hspost}@{}}%
\column{E}{@{}>{\hspre}l<{\hspost}@{}}%
\>[B]{}\Varid{d2}\mathrel{=}\Varid{uniform}\;(\Varid{words}\;\text{\ttfamily \char34 Uma~frase~de~cinco~palavras\char34}){}\<[E]%
\ColumnHook
\end{hscode}\resethooks
isto é
\begin{Verbatim}[fontsize=\small]
     "Uma"  20.0%
   "cinco"  20.0%
      "de"  20.0%
   "frase"  20.0%
"palavras"  20.0%
\end{Verbatim}
distribuição \emph{normais}, eg.\
\begin{hscode}\SaveRestoreHook
\column{B}{@{}>{\hspre}l<{\hspost}@{}}%
\column{E}{@{}>{\hspre}l<{\hspost}@{}}%
\>[B]{}\Varid{d3}\mathrel{=}\Varid{normal}\;[\mskip1.5mu \mathrm{10}\mathinner{\ldotp\ldotp}\mathrm{20}\mskip1.5mu]{}\<[E]%
\ColumnHook
\end{hscode}\resethooks
etc.\footnote{Para mais detalhes ver o código fonte de \Probability, que é uma adaptação da
biblioteca \PFP\ (``Probabilistic Functional Programming''). Para quem quiser saber mais
recomenda-se a leitura do artigo \cite{EK06}.}
\ensuremath{\fun{Dist}} forma um \textbf{mónade} cuja unidade é \ensuremath{\Varid{return}\;\Varid{a}\mathrel{=}\Conid{D}\;[\mskip1.5mu (\Varid{a},\mathrm{1})\mskip1.5mu]} e cuja composição de Kleisli
é (simplificando a notação)
\begin{hscode}\SaveRestoreHook
\column{B}{@{}>{\hspre}l<{\hspost}@{}}%
\column{3}{@{}>{\hspre}l<{\hspost}@{}}%
\column{E}{@{}>{\hspre}l<{\hspost}@{}}%
\>[3]{}(\Varid{f}\kcomp \Varid{g})\;\Varid{a}\mathrel{=}[\mskip1.5mu (\Varid{y},\Varid{q}\mathbin{*}\Varid{p})\mid (\Varid{x},\Varid{p})\leftarrow \Varid{g}\;\Varid{a},(\Varid{y},\Varid{q})\leftarrow \Varid{f}\;\Varid{x}\mskip1.5mu]{}\<[E]%
\ColumnHook
\end{hscode}\resethooks
em que \ensuremath{\Varid{g}\mathbin{:}\Conid{A}\to \fun{Dist}\;\mathit B} e \ensuremath{\Varid{f}\mathbin{:}\mathit B\to \fun{Dist}\;\mathit C} são funções \textbf{monádicas} que representam
\emph{computações probabilísticas}.

Este mónade é adequado à resolução de problemas de \emph{probabilidades e estatística} usando programação funcional, de forma elegante e como caso particular da programação monádica.

\section{Código fornecido}\label{sec:codigo}

\subsection*{Problema 1}

Árvores exemplo:
\begin{hscode}\SaveRestoreHook
\column{B}{@{}>{\hspre}l<{\hspost}@{}}%
\column{3}{@{}>{\hspre}l<{\hspost}@{}}%
\column{4}{@{}>{\hspre}l<{\hspost}@{}}%
\column{5}{@{}>{\hspre}l<{\hspost}@{}}%
\column{12}{@{}>{\hspre}l<{\hspost}@{}}%
\column{E}{@{}>{\hspre}l<{\hspost}@{}}%
\>[B]{}t_1 \mathbin{::}\Conid{BTree}\;\Conid{Int}{}\<[E]%
\\
\>[B]{}t_1 \mathrel{=}\Conid{Node}\;(\mathrm{5},(\Conid{Node}\;(\mathrm{3},(\Conid{Node}\;(\mathrm{1},(\Conid{Empty},\Conid{Empty})),\Conid{Node}\;(\mathrm{4},(\Conid{Empty},\Conid{Empty})))),{}\<[E]%
\\
\>[B]{}\hsindent{12}{}\<[12]%
\>[12]{}\Conid{Node}\;(\mathrm{7},(\Conid{Node}\;(\mathrm{6},(\Conid{Empty},\Conid{Empty})),\Conid{Node}\;(\mathrm{8},(\Conid{Empty},\Conid{Empty})))))){}\<[E]%
\\[\blanklineskip]%
\>[B]{}t_2 \mathbin{::}\Conid{BTree}\;\Conid{Int}{}\<[E]%
\\
\>[B]{}t_2 \mathrel{=}{}\<[E]%
\\
\>[B]{}\hsindent{3}{}\<[3]%
\>[3]{}\Varid{node}\;\mathrm{1}\;{}\<[E]%
\\
\>[3]{}\hsindent{2}{}\<[5]%
\>[5]{}(\Varid{node}\;\mathrm{2}\;(\Varid{node}\;\mathrm{4}\;\Conid{Empty}\;\Conid{Empty})\;(\Varid{node}\;\mathrm{5}\;\Conid{Empty}\;\Conid{Empty}))\;{}\<[E]%
\\
\>[3]{}\hsindent{2}{}\<[5]%
\>[5]{}(\Varid{node}\;\mathrm{3}\;(\Varid{node}\;\mathrm{6}\;\Conid{Empty}\;\Conid{Empty})\;(\Varid{node}\;\mathrm{7}\;\Conid{Empty}\;\Conid{Empty})){}\<[E]%
\\[\blanklineskip]%
\>[B]{}t_3 \mathbin{::}\Conid{BTree}\;\Conid{Char}{}\<[E]%
\\
\>[B]{}t_3 \mathrel{=}{}\<[E]%
\\
\>[B]{}\hsindent{3}{}\<[3]%
\>[3]{}\Varid{node}\;\text{\ttfamily 'A'}\;{}\<[E]%
\\
\>[3]{}\hsindent{2}{}\<[5]%
\>[5]{}(\Varid{node}\;\text{\ttfamily 'B'}\;(\Varid{node}\;\text{\ttfamily 'C'}\;(\Varid{node}\;\text{\ttfamily 'D'}\;\Conid{Empty}\;\Conid{Empty})\;\Conid{Empty})\;\Conid{Empty})\;{}\<[E]%
\\
\>[3]{}\hsindent{2}{}\<[5]%
\>[5]{}(\Varid{node}\;\text{\ttfamily 'E'}\;\Conid{Empty}\;\Conid{Empty}){}\<[E]%
\\[\blanklineskip]%
\>[B]{}t_4 \mathbin{::}\Conid{BTree}\;\Conid{Char}{}\<[E]%
\\
\>[B]{}t_4 \mathrel{=}{}\<[E]%
\\
\>[B]{}\hsindent{3}{}\<[3]%
\>[3]{}\Varid{node}\;\text{\ttfamily 'A'}\;{}\<[E]%
\\
\>[3]{}\hsindent{2}{}\<[5]%
\>[5]{}(\Varid{node}\;\text{\ttfamily 'B'}\;(\Varid{node}\;\text{\ttfamily 'C'}\;(\Varid{node}\;\text{\ttfamily 'D'}\;\Conid{Empty}\;\Conid{Empty})\;\Conid{Empty})\;\Conid{Empty})\;{}\<[E]%
\\
\>[3]{}\hsindent{2}{}\<[5]%
\>[5]{}\Conid{Empty}{}\<[E]%
\\[\blanklineskip]%
\>[B]{}t_5 \mathbin{::}\Conid{BTree}\;\Conid{Int}{}\<[E]%
\\
\>[B]{}t_5 \mathrel{=}{}\<[E]%
\\
\>[B]{}\hsindent{3}{}\<[3]%
\>[3]{}\Varid{node}\;\mathrm{1}\;{}\<[E]%
\\
\>[3]{}\hsindent{1}{}\<[4]%
\>[4]{}(\Varid{node}\;\mathrm{2}\;(\Varid{node}\;\mathrm{4}\;\Conid{Empty}\;\Conid{Empty})\;\Conid{Empty})\;{}\<[E]%
\\
\>[3]{}\hsindent{1}{}\<[4]%
\>[4]{}(\Varid{node}\;\mathrm{3}\;\Conid{Empty}\;(\Varid{node}\;\mathrm{5}\;(\Varid{node}\;\mathrm{6}\;\Conid{Empty}\;\Conid{Empty})\;\Conid{Empty})){}\<[E]%
\\[\blanklineskip]%
\>[B]{}\Varid{node}\;\Varid{a}\;\Varid{b}\;\Varid{c}\mathrel{=}\Conid{Node}\;(\Varid{a},(\Varid{b},\Varid{c})){}\<[E]%
\ColumnHook
\end{hscode}\resethooks

%----------------- Soluções dos alunos -----------------------------------------%

\section{Soluções dos alunos}\label{sec:resolucao}
Os alunos devem colocar neste anexo as suas soluções para os exercícios
propostos, de acordo com o ``layout'' que se fornece.
Não podem ser alterados os nomes ou tipos das funções dadas, mas pode ser
adicionado texto ao anexo, bem como diagramas e/ou outras funções auxiliares
que sejam necessárias.

\noindent
\textbf{Importante}: Não pode ser alterado o texto deste ficheiro fora deste anexo.

\subsection*{Problema 1}
\textbf{Catamorfismo}

Na primeira versão proposta para a resolução do \textbf{Problema 1}, pretende-se
usar um catamorfismo de \ensuremath{\Conid{BTrees}} para se fazer a travessia \emph{in-order} em regime \emph{breadth-first}.

Se repararmos, o resultado de aplicarmos a função \ensuremath{\Varid{levels}} a uma \ensuremath{\Conid{BTree}} é uma lista de listas,
onde cada uma dessas listas internas corresponde aos valores dos nós de um nível da árvore. Ou seja,
a aplicação de \ensuremath{\Varid{levels}} à árvore \ensuremath{t_1 }, por exemplo, resulta na lista de listas:

\begin{hscode}\SaveRestoreHook
\column{B}{@{}>{\hspre}l<{\hspost}@{}}%
\column{E}{@{}>{\hspre}l<{\hspost}@{}}%
\>[B]{}[\mskip1.5mu [\mskip1.5mu \mathrm{5}\mskip1.5mu],[\mskip1.5mu \mathrm{3},\mathrm{7}\mskip1.5mu],[\mskip1.5mu \mathrm{1},\mathrm{4},\mathrm{6},\mathrm{8}\mskip1.5mu]\mskip1.5mu]{}\<[E]%
\ColumnHook
\end{hscode}\resethooks

Para depois obter a travessia \ensuremath{\Varid{bforder}}, basta concatenar todas as listas internas, o que é feito
na função \ensuremath{\Varid{bfsLevels}} com recurso à função \ensuremath{\Varid{concat}}.

O desafio aqui está em encontrar o gene (\ensuremath{\Varid{glevels}}) do catamorfismo \ensuremath{\Varid{levels}}. 
Para começar, podemos representar esse catamorfismo através do seguinte diagrama:

\begin{eqnarray*}
\centerline{
     \xymatrix@C=5cm@R=2cm{
          \ensuremath{\Conid{BTree}\;\Conid{A}}
               \ar[d]_-{\ensuremath{\Varid{levels}\mathrel{=}\llparenthesis\, \Varid{glevels}\,\rrparenthesis}}
               \ar@/^1pc/[r]^-{\ensuremath{\Varid{outBTree}}}
     &
          \ensuremath{\mathrm{1}\mathbin{+}\Conid{A}\times(\Conid{BTree}\;\Conid{A}\times\Conid{BTree}\;\Conid{A})}
               \ar[d]^{\ensuremath{\Varid{id}\mathbin{+}\Varid{id}\times(\Varid{levels}\times\Varid{levels})}}
               \ar@/^1pc/[l]^-{\ensuremath{\Varid{inBTree}}}
     \\
          \ensuremath{{({\Conid{A}}^{*})}^{*}}
     &
          \ensuremath{\mathrm{1}\mathbin{+}\Conid{A}\times({({\Conid{A}}^{*})}^{*}\times{({\Conid{A}}^{*})}^{*})}
               \ar@/^1pc/[l]^-{\ensuremath{\Varid{glevels}}}
     }
}
\end{eqnarray*}

A partir deste diagrama, percebemos que o gene do catamorfismo deverá, para obter a tal lista de listas,
colocar o valor da raiz no início da lista final, seguido das listas dos níveis das subárvores esquerda e direita.
Contudo essas listas dos níveis das subárvores estão também organizadas por níveis, ou seja,
a primeira lista corresponde ao nível 1, a segunda ao nível 2, etc. No caso da àrvore \ensuremath{t_1 }, por exemplo,
a aplicação de \ensuremath{\Varid{levels}} às subárvores esquerda e direita de \ensuremath{t_1 } resulta, respectivamente, nas listas de listas:
\begin{hscode}\SaveRestoreHook
\column{B}{@{}>{\hspre}l<{\hspost}@{}}%
\column{E}{@{}>{\hspre}l<{\hspost}@{}}%
\>[B]{}[\mskip1.5mu [\mskip1.5mu \mathrm{3}\mskip1.5mu],[\mskip1.5mu \mathrm{1},\mathrm{4}\mskip1.5mu]\mskip1.5mu]\;\Varid{e}\;[\mskip1.5mu [\mskip1.5mu \mathrm{7}\mskip1.5mu],[\mskip1.5mu \mathrm{6},\mathrm{8}\mskip1.5mu]\mskip1.5mu]{}\<[E]%
\ColumnHook
\end{hscode}\resethooks

Neste caso o passo final seria juntar as listas dos níveis das subárvores esquerda e direita, 
e colocar o valor da raiz \ensuremath{\mathrm{5}} (\ensuremath{[\mskip1.5mu \mathrm{5}\mskip1.5mu]}) no início da lista final. Para juntar as listas 
dos níveis das subárvores esquerda e direita, podemos definir a seguinte função auxiliar:

\begin{hscode}\SaveRestoreHook
\column{B}{@{}>{\hspre}l<{\hspost}@{}}%
\column{E}{@{}>{\hspre}l<{\hspost}@{}}%
\>[B]{}\Varid{juntaListas}\mathbin{::}([\mskip1.5mu [\mskip1.5mu \Varid{a}\mskip1.5mu]\mskip1.5mu],[\mskip1.5mu [\mskip1.5mu \Varid{a}\mskip1.5mu]\mskip1.5mu])\to [\mskip1.5mu [\mskip1.5mu \Varid{a}\mskip1.5mu]\mskip1.5mu]{}\<[E]%
\\
\>[B]{}\Varid{juntaListas}\;(\Varid{l},[\mskip1.5mu \mskip1.5mu])\mathrel{=}\Varid{l}{}\<[E]%
\\
\>[B]{}\Varid{juntaListas}\;([\mskip1.5mu \mskip1.5mu],\Varid{r})\mathrel{=}\Varid{r}{}\<[E]%
\\
\>[B]{}\Varid{juntaListas}\;((\Varid{a}\mathbin{:}\Varid{as}),(\Varid{b}\mathbin{:}\Varid{bs}))\mathrel{=}(\Varid{a}\mathbin{+\!\!+}\Varid{b})\mathbin{:}\Varid{juntaListas}\;(\Varid{as},\Varid{bs}){}\<[E]%
\ColumnHook
\end{hscode}\resethooks

Para o exemplo da árvore \ensuremath{t_1 }, a aplicação de \ensuremath{\Varid{juntaListas}} às listas de listas
\begin{hscode}\SaveRestoreHook
\column{B}{@{}>{\hspre}l<{\hspost}@{}}%
\column{E}{@{}>{\hspre}l<{\hspost}@{}}%
\>[B]{}[\mskip1.5mu [\mskip1.5mu \mathrm{3}\mskip1.5mu],[\mskip1.5mu \mathrm{1},\mathrm{4}\mskip1.5mu]\mskip1.5mu]\;\Varid{e}\;[\mskip1.5mu [\mskip1.5mu \mathrm{7}\mskip1.5mu],[\mskip1.5mu \mathrm{6},\mathrm{8}\mskip1.5mu]\mskip1.5mu]{}\<[E]%
\ColumnHook
\end{hscode}\resethooks
resulta na lista de listas:
\begin{hscode}\SaveRestoreHook
\column{B}{@{}>{\hspre}l<{\hspost}@{}}%
\column{E}{@{}>{\hspre}l<{\hspost}@{}}%
\>[B]{}[\mskip1.5mu [\mskip1.5mu \mathrm{3},\mathrm{7}\mskip1.5mu],[\mskip1.5mu \mathrm{1},\mathrm{4},\mathrm{6},\mathrm{8}\mskip1.5mu]\mskip1.5mu]{}\<[E]%
\ColumnHook
\end{hscode}\resethooks

Assim sendo, o comportamento do gene \ensuremath{\Varid{glevels}} pode ser representado por este diagrama:

\begin{eqnarray*}
\centerline{
     \xymatrix@C=2cm@R=2cm{
     &
          \ensuremath{{({\Conid{A}}^{*})}^{*}}
     &
     \\
          \ensuremath{\mathrm{1}}
               \ar[ur]^-{\ensuremath{\Varid{nil}}}
               \ar[r]_-{\ensuremath{i_1}}
     &
          \ensuremath{\mathrm{1}\mathbin{+}\Conid{A}\times({({\Conid{A}}^{*})}^{*}\times{({\Conid{A}}^{*})}^{*})}
               \ar[u]^-{\ensuremath{\alt{\Varid{nil}}{\Varid{f}}}}
     &
          \ensuremath{\Conid{A}\times({({\Conid{A}}^{*})}^{*}\times{({\Conid{A}}^{*})}^{*})}
               \ar[l]^-{\ensuremath{i_2}}
               \ar[ul]_-{\ensuremath{\Varid{f}}}
     }
}
\end{eqnarray*}

E o código que define o gene \ensuremath{\Varid{glevels}} é o seguinte:

\begin{hscode}\SaveRestoreHook
\column{B}{@{}>{\hspre}l<{\hspost}@{}}%
\column{6}{@{}>{\hspre}l<{\hspost}@{}}%
\column{E}{@{}>{\hspre}l<{\hspost}@{}}%
\>[B]{}\Varid{glevels}\mathbin{::}()+(\Varid{a},([\mskip1.5mu [\mskip1.5mu \Varid{a}\mskip1.5mu]\mskip1.5mu],[\mskip1.5mu [\mskip1.5mu \Varid{a}\mskip1.5mu]\mskip1.5mu]))\to [\mskip1.5mu [\mskip1.5mu \Varid{a}\mskip1.5mu]\mskip1.5mu]{}\<[E]%
\\
\>[B]{}\Varid{glevels}\mathrel{=}\alt{\Varid{nil}}{\Varid{f}}{}\<[E]%
\\
\>[B]{}\hsindent{6}{}\<[6]%
\>[6]{}\mathbf{where}\;\Varid{f}\;(\Varid{a},(\Varid{l},\Varid{r}))\mathrel{=}[\mskip1.5mu \Varid{a}\mskip1.5mu]\mathbin{:}\Varid{juntaListas}\;(\Varid{l},\Varid{r}){}\<[E]%
\ColumnHook
\end{hscode}\resethooks

Tal como descrito anteriormente, \ensuremath{\Varid{f}} coloca o valor da raiz à cabeça da lista,
seguido da concatenação das listas dos níveis das subárvores esquerda e direita,
com a função \ensuremath{\Varid{juntaListas}}.

Se quisermos definir \ensuremath{\Varid{glevels}} numa versão completamente \ensuremath{\Varid{pointfree}}, 
podemos esquematizar o lado direito da solução desta forma:


\begin{eqnarray*}
\centerline{
\xymatrix@R=1cm{
   \ensuremath{\Conid{A}\times({({\Conid{A}}^{*})}^{*}\times{({\Conid{A}}^{*})}^{*})}
      \ar[d]^-{\ensuremath{\Varid{singl}\times\Varid{juntaListas}}} 
\\
   \ensuremath{{\Conid{A}}^{*}\times{({\Conid{A}}^{*})}^{*}}
      \ar[d]^-{\ensuremath{\Varid{cons}}}
\\
   \ensuremath{{({\Conid{A}}^{*})}^{*}}
}
}
\end{eqnarray*}

Logo a definição \ensuremath{\Varid{pointfree}} de \ensuremath{\Varid{glevels}} é a seguinte:
\begin{hscode}\SaveRestoreHook
\column{B}{@{}>{\hspre}l<{\hspost}@{}}%
\column{E}{@{}>{\hspre}l<{\hspost}@{}}%
\>[B]{}\Varid{glevels'}\mathrel{=}\alt{\Varid{nil}}{\Varid{cons}\comp (\Varid{singl}\times\Varid{juntaListas})}{}\<[E]%
\ColumnHook
\end{hscode}\resethooks

\textbf{Anamorfismo}

Na segunda versão proposta para a resolução do \textbf{Problema 1}, pretende-se
usar um anamorfismo de listas para se fazer a travessia \emph{in-order} em regime \emph{breadth-first}.

Com o estudo do artigo \cite{Ok00}, percebemos que é possível fazer a travessia \emph{breadth-first}
em \ensuremath{\Conid{BTrees}} usando uma floresta para representar os nós a visitar
(neste caso uma floresta é uma lista de \ensuremath{\Conid{BTrees}}).

A ideia do algoritmo é visitar os nós da floresta, retirando o valor do nó da frente da lista,
adicionando o seu valor à lista resultado (caso não seja \ensuremath{\Conid{Empty}}) e 
adicionando à floresta os filhos esquerdo e direito do nó visitado (caso não seja \ensuremath{\Conid{Empty}}), 
mas colocando-os no final da lista. Este processo repete-se até que a floresta esteja vazia.

Para representar este algoritmo como um anamorfismo, começamos por desenhar o seguinte diagrama:

\begin{eqnarray*}
\centerline{
     \xymatrix@C=5cm@R=2cm{
          \ensuremath{{\Conid{A}}^{*}}
               \ar@/^1pc/[r]^-{\ensuremath{\Varid{outList}}}
     &
          \ensuremath{\mathrm{1}\mathbin{+}\Conid{A}\times{\Conid{A}}^{*}}
               \ar@/^1pc/[l]^-{\ensuremath{\Varid{inList}}}
     \\
          \ensuremath{{(\Conid{BTree}\;\Conid{A})}^{*}}
               \ar[u]^-{\ensuremath{\anaList{\Varid{geneBF}}}}
               \ar@/_1pc/[r]_-{\ensuremath{\Varid{geneBF}}}
     &
          \ensuremath{\mathrm{1}\mathbin{+}\Conid{A}\times{(\Conid{BTree}\;\Conid{A})}^{*}}
               \ar[u]_{\ensuremath{\Varid{id}\mathbin{+}\Varid{id}\times\anaList{\Varid{geneBF}}}}
     }
}
\end{eqnarray*}

Contudo, como referido anteriormente, o algoritmo usa uma floresta (lista de \ensuremath{\Conid{BTrees}}), e não um \ensuremath{\Conid{BTree}}, 
que é o input da função \ensuremath{\Varid{bft}}. Desta forma é necesário garantir que a \ensuremath{\Conid{BTree}} argumento seja convertida
numa floresta, para poder ser usada no anamorfismo. Para isso, usamos a função \ensuremath{\Varid{singl}},
que transforma uma \ensuremath{\Conid{BTree}} numa floresta com apenas essa \ensuremath{\Conid{BTree}} (lista com apenas um elemento).

\begin{eqnarray*}
\centerline{
     \xymatrix@C=3cm{
          \ensuremath{\Conid{BTree}\;\Conid{A}}
               \ar[r]^-{\ensuremath{\Varid{singl}}}
     &
          \ensuremath{{(\Conid{BTree}\;\Conid{A})}^{*}}
     }
}
\end{eqnarray*}

Assim sendo, a função \ensuremath{\Varid{bft}} é definida como a composição do anamorfismo \ensuremath{\anaList{\Varid{geneBF}}}
com a função \ensuremath{\Varid{singl}}, como se mostra a seguir:

\begin{hscode}\SaveRestoreHook
\column{B}{@{}>{\hspre}l<{\hspost}@{}}%
\column{E}{@{}>{\hspre}l<{\hspost}@{}}%
\>[B]{}\Varid{bft}\mathrel{=}\anaList{\Varid{geneBF}}\comp \Varid{singl}{}\<[E]%
\ColumnHook
\end{hscode}\resethooks

Para o gene do anamorfismo, o comportamento é o já anteriormente descrito,
ou seja, visitar os nós da floresta, retirando o valor do nó da frente da lista, 
e devolvendo um par com o valor do nó e a floresta a visitar. Na floresta a visitar,
já estarão adicionados os filhos esquerdo e direito do nó visitado (caso não seja \ensuremath{\Conid{Empty}}),
colocados no final da lista.
O anamorfismo repete este processo até que a floresta esteja vazia.

\begin{hscode}\SaveRestoreHook
\column{B}{@{}>{\hspre}l<{\hspost}@{}}%
\column{E}{@{}>{\hspre}l<{\hspost}@{}}%
\>[B]{}\Varid{geneBF}\mathbin{::}[\mskip1.5mu \Conid{BTree}\;\Varid{a}\mskip1.5mu]\to ()+(\Varid{a},[\mskip1.5mu \Conid{BTree}\;\Varid{a}\mskip1.5mu]){}\<[E]%
\\
\>[B]{}\Varid{geneBF}\;[\mskip1.5mu \mskip1.5mu]\mathrel{=}i_1\;(){}\<[E]%
\\
\>[B]{}\Varid{geneBF}\;(\Conid{Empty}\mathbin{:}\Varid{t})\mathrel{=}\Varid{geneBF}\;\Varid{t}{}\<[E]%
\\
\>[B]{}\Varid{geneBF}\;(\Conid{Node}\;(\Varid{a},(\Varid{l},\Varid{r}))\mathbin{:}\Varid{t})\mathrel{=}i_2\;(\Varid{a},\Varid{t}\mathbin{+\!\!+}[\mskip1.5mu \Varid{l},\Varid{r}\mskip1.5mu]){}\<[E]%
\ColumnHook
\end{hscode}\resethooks

\subsection*{Problema 2}
No \textbf{Problema 2}, 


\subsection*{Problema 3}
No \textbf{Problema 3}, pretende-se definir a função \ensuremath{\Varid{fair\char95 merge}} como um anamorfismo de \ensuremath{\Conid{Streams}}.
Esta estratégia de \ensuremath{\Varid{merging}} permite que os elementos das duas \ensuremath{\Conid{Streams}} sejam intercalados de forma justa,
ou seja, cada elemento de uma \ensuremath{\Conid{Stream}} é seguido por um elemento da outra \ensuremath{\Conid{Stream}}.

Tal como pedido no enunciado, vamos começar por derivar a lei dual da recursividade mútua.

\begin{eqnarray*}
\start
\ensuremath{\alt{\Varid{f}}{\Varid{g}}\mathrel{=}\lanabracket\,\alt{\Varid{h}}{\Varid{k}}\,\ranabracket}
\just\equiv{ Universal-ana }
\ensuremath{\mathsf{out}\comp \alt{\Varid{f}}{\Varid{g}}\mathrel{=}\fun F \;\alt{\Varid{f}}{\Varid{g}}\comp \alt{\Varid{h}}{\Varid{k}}}
\just\equiv{ Fusão- + (2x)}
\ensuremath{\alt{\mathsf{out}\comp \Varid{f}}{\mathsf{out}\comp \Varid{g}}\mathrel{=}\alt{\fun F \;\alt{\Varid{f}}{\Varid{g}}\comp \Varid{h}}{\fun F \;\alt{\Varid{f}}{\Varid{g}}\comp \Varid{k}}}
\just\equiv{ Eq- + }
\ensuremath{\begin{lcbr}\mathsf{out}\comp \Varid{f}\mathrel{=}\fun F \;\alt{\Varid{f}}{\Varid{g}}\comp \Varid{h}\\\mathsf{out}\comp \Varid{g}\mathrel{=}\fun F \;\alt{\Varid{f}}{\Varid{g}}\comp \Varid{k}\end{lcbr}}
\qed
\end{eqnarray*}

Fica então demonstrada a lei dual da recursividade mútua.

A partir da definição do tipo \ensuremath{\Conid{Stream}} e da função \ensuremath{\mathsf{out}}, é possível 
deduzir que o functor \ensuremath{\fun F } associado a \ensuremath{\Conid{Stream}} é o seguinte,

\begin{hscode}\SaveRestoreHook
\column{B}{@{}>{\hspre}l<{\hspost}@{}}%
\column{E}{@{}>{\hspre}l<{\hspost}@{}}%
\>[B]{}\fun F \;\Conid{X}\mathrel{=}\Conid{A}\times\Conid{X}{}\<[E]%
\\
\>[B]{}\fun F \;\Varid{g}\mathrel{=}\Varid{id}\times\Varid{g}{}\<[E]%
\ColumnHook
\end{hscode}\resethooks

já que uma \ensuremath{\Conid{Stream}\;\Conid{A}} é decomposta num par com o primeiro elemento de um tipo qualquer \ensuremath{\Conid{A}} e o resto da \ensuremath{\Conid{Stream}\;\Conid{A}}.

Tendo isto em conta, e sabendo que a função \ensuremath{\Varid{fair\char95 merge'}} que queremos definir como um anamorfismo de \ensuremath{\Conid{Streams}} 
é do tipo
\begin{hscode}\SaveRestoreHook
\column{B}{@{}>{\hspre}l<{\hspost}@{}}%
\column{E}{@{}>{\hspre}l<{\hspost}@{}}%
\>[B]{}\Varid{fair\char95 merge'}\mathbin{::}(\Conid{Stream}\;\Varid{a},\Conid{Stream}\;\Varid{a})+(\Conid{Stream}\;\Varid{a},\Conid{Stream}\;\Varid{a})\to \Conid{Stream}\;\Varid{a}{}\<[E]%
\ColumnHook
\end{hscode}\resethooks
podemos representar o anamorfismo \ensuremath{\Varid{fair\char95 merge'}} através do seguinte diagrama:

\begin{eqnarray*}
\centerline{
     \xymatrix@C=1.5cm@R=3.5cm{
          \ensuremath{\Conid{Stream}\;\Conid{A}}
               \ar[r]^-{\ensuremath{\mathsf{out}}}
     &
          \ensuremath{\Conid{A}\times\Conid{Stream}\;\Conid{A}}
     \\
          \ensuremath{(\Conid{Stream}\;\Conid{A}\times\Conid{Stream}\;\Conid{A})+(\Conid{Stream}\;\Conid{A}\times\Conid{Stream}\;\Conid{A})}
               \ar[u]^-{\ensuremath{\lanabracket\,\Varid{geneFM}\,\ranabracket}}
               \ar@/_1pc/[r]_-{\ensuremath{\Varid{geneFM}}}
     &
          \ensuremath{\Conid{A}\times((\Conid{Stream}\;\Conid{A}\times\Conid{Stream}\;\Conid{A})+(\Conid{Stream}\;\Conid{A}\times\Conid{Stream}\;\Conid{A}))}
               \ar[u]_{\ensuremath{\fun F \;\lanabracket\,\Varid{geneFM}\,\ranabracket\mathrel{=}\Varid{id}\times\lanabracket\,\Varid{geneFM}\,\ranabracket}}
     }
}
\end{eqnarray*}

Com a análise deste diagrama, e do comportamento da função \ensuremath{\Varid{fair\char95 merge}} definida 
de forma mutuamente recursiva na formulação do problema neste enunciado
\begin{hscode}\SaveRestoreHook
\column{B}{@{}>{\hspre}l<{\hspost}@{}}%
\column{4}{@{}>{\hspre}l<{\hspost}@{}}%
\column{E}{@{}>{\hspre}l<{\hspost}@{}}%
\>[B]{}\Varid{fair\char95 merge}\mathbin{::}(\Conid{Stream}\;\Varid{a},\Conid{Stream}\;\Varid{a})+(\Conid{Stream}\;\Varid{a},\Conid{Stream}\;\Varid{a})\to \Conid{Stream}\;\Varid{a}{}\<[E]%
\\
\>[B]{}\Varid{fair\char95 merge}\mathrel{=}\alt{\Varid{h}}{\Varid{k}}\;\mathbf{where}{}\<[E]%
\\
\>[B]{}\hsindent{4}{}\<[4]%
\>[4]{}\Varid{h}\;(\Conid{Cons}\;(\Varid{x},\Varid{xs}),\Varid{y})\mathrel{=}\Conid{Cons}\;(\Varid{x},\Varid{k}\;(\Varid{xs},\Varid{y})){}\<[E]%
\\
\>[B]{}\hsindent{4}{}\<[4]%
\>[4]{}\Varid{k}\;(\Varid{x},\Conid{Cons}\;(\Varid{y},\Varid{ys}))\mathrel{=}\Conid{Cons}\;(\Varid{y},\Varid{h}\;(\Varid{x},\Varid{ys})){}\<[E]%
\ColumnHook
\end{hscode}\resethooks

conseguimos definir \ensuremath{\Varid{fair\char95 merge'}} como um anamorfismo de \ensuremath{\Conid{Streams}}
\begin{hscode}\SaveRestoreHook
\column{B}{@{}>{\hspre}l<{\hspost}@{}}%
\column{E}{@{}>{\hspre}l<{\hspost}@{}}%
\>[B]{}\Varid{fair\char95 merge'}\mathrel{=}\lanabracket\,\Varid{geneFM}\,\ranabracket{}\<[E]%
\ColumnHook
\end{hscode}\resethooks

onde o gene \ensuremath{\Varid{geneFM}} é definido como:

\begin{hscode}\SaveRestoreHook
\column{B}{@{}>{\hspre}l<{\hspost}@{}}%
\column{E}{@{}>{\hspre}l<{\hspost}@{}}%
\>[B]{}\Varid{geneFM}\mathbin{::}(\Conid{Stream}\;\Varid{a},\Conid{Stream}\;\Varid{a})+(\Conid{Stream}\;\Varid{a},\Conid{Stream}\;\Varid{a})\to (\Varid{a},(\Conid{Stream}\;\Varid{a},\Conid{Stream}\;\Varid{a})+(\Conid{Stream}\;\Varid{a},\Conid{Stream}\;\Varid{a})){}\<[E]%
\\
\>[B]{}\Varid{geneFM}\;(i_1\;(\Conid{Cons}\;(\Varid{x},\Varid{xs}),\Varid{y}))\mathrel{=}(\Varid{x},i_2\;(\Varid{xs},\Varid{y})){}\<[E]%
\\
\>[B]{}\Varid{geneFM}\;(i_2\;(\Varid{x},\Conid{Cons}\;(\Varid{y},\Varid{ys})))\mathrel{=}(\Varid{y},i_1\;(\Varid{x},\Varid{ys})){}\<[E]%
\ColumnHook
\end{hscode}\resethooks

Para comprovar este resultado, podemos utilizar a lei dual da recursividade mútua que foi demonstrada anteriormente.
Seja \ensuremath{\Varid{fair\char95 merge}} dada por:
\begin{hscode}\SaveRestoreHook
\column{B}{@{}>{\hspre}l<{\hspost}@{}}%
\column{E}{@{}>{\hspre}l<{\hspost}@{}}%
\>[B]{}\Varid{fair\char95 merge}\mathrel{=}\alt{\Varid{h}}{\Varid{k}}{}\<[E]%
\ColumnHook
\end{hscode}\resethooks

E querendo demonstrar que \ensuremath{\Varid{fair\char95 merge}\mathrel{=}\lanabracket\,\Varid{geneFM}\,\ranabracket}, onde \ensuremath{\Varid{geneFM}} pode 
ser decomposto em:
\begin{hscode}\SaveRestoreHook
\column{B}{@{}>{\hspre}l<{\hspost}@{}}%
\column{E}{@{}>{\hspre}l<{\hspost}@{}}%
\>[B]{}\Varid{geneFM}\mathrel{=}\alt{\Varid{g1}}{\Varid{g2}}{}\<[E]%
\ColumnHook
\end{hscode}\resethooks

Tem-se:

\begin{eqnarray*}
\start
\ensuremath{\alt{\Varid{h}}{\Varid{k}}\mathrel{=}\lanabracket\,\Varid{geneFM}\,\ranabracket}
\just\equiv{ Lei dual da recursividade mútua }
\ensuremath{\begin{lcbr}\mathsf{out}\comp \Varid{h}\mathrel{=}(\Varid{id}\times\alt{\Varid{h}}{\Varid{k}})\comp \Varid{g1}\\\mathsf{out}\comp \Varid{k}\mathrel{=}(\Varid{id}\times\alt{\Varid{h}}{\Varid{k}})\comp \Varid{g2}\end{lcbr}}
\qed
\end{eqnarray*}

Para que estas equações sejam verdadeiras, é necessário encontrar \ensuremath{\Varid{g1}} e \ensuremath{\Varid{g2}}
tais que as igualdades se verifiquem.

Vamos definir \ensuremath{\Varid{g1}} e \ensuremath{\Varid{g2}} como:

\begin{hscode}\SaveRestoreHook
\column{B}{@{}>{\hspre}l<{\hspost}@{}}%
\column{E}{@{}>{\hspre}l<{\hspost}@{}}%
\>[B]{}\Varid{g1}\;(\Conid{Cons}\;(\Varid{x},\Varid{xs}),\Varid{y})\mathrel{=}(\Varid{x},i_2\;(\Varid{xs},\Varid{y})){}\<[E]%
\\[\blanklineskip]%
\>[B]{}\Varid{g2}\;(\Varid{x},\Conid{Cons}\;(\Varid{y},\Varid{ys}))\mathrel{=}(\Varid{y},i_1\;(\Varid{x},\Varid{ys})){}\<[E]%
\ColumnHook
\end{hscode}\resethooks

Para a primeira equação, temos:
\begin{eqnarray*}
\start
\just\equiv{ Igualdade extensional }
\ensuremath{\mathsf{out}\comp \Varid{h}\;(\Conid{Cons}\;(\Varid{x},\Varid{xs}),\Varid{y})}
\just\equiv{ Def-comp }
\ensuremath{\mathsf{out}\;(\Varid{h}\;(\Conid{Cons}\;(\Varid{x},\Varid{xs}),\Varid{y}))}
\just\equiv{ Def-h }
\ensuremath{\mathsf{out}\;(\Conid{Cons}\;(\Varid{x},\Varid{k}\;(\Varid{xs},\Varid{y})))}
\just\equiv{ Def-out }
\ensuremath{(\Varid{x},\Varid{k}\;(\Varid{xs},\Varid{y}))}
\just\equiv{ Def-x, Cancelamento-+ }
\ensuremath{(\Varid{id}\times\alt{\Varid{h}}{\Varid{k}})\;(\Varid{x},i_2\;(\Varid{xs},\Varid{y}))}
\just\equiv{ Def-g1 }
\ensuremath{(\Varid{id}\times\alt{\Varid{h}}{\Varid{k}})\comp \Varid{g1}\;(\Conid{Cons}\;(\Varid{x},\Varid{xs}),\Varid{y})}
\qed
\end{eqnarray*}

O mesmo aplica-se para a segunda equação:
\begin{eqnarray*}
\start
\just\equiv{ Igualdade extensional }
\ensuremath{\mathsf{out}\comp \Varid{k}\;(\Varid{x},\Conid{Cons}\;(\Varid{y},\Varid{ys}))}
\just\equiv{ Def-comp }
\ensuremath{\mathsf{out}\;(\Varid{k}\;(\Varid{x},\Conid{Cons}\;(\Varid{y},\Varid{ys})))}
\just\equiv{ Def-k }
\ensuremath{\mathsf{out}\;(\Conid{Cons}\;(\Varid{y},\Varid{h}\;(\Varid{x},\Varid{ys})))}
\just\equiv{ Def-out }
\ensuremath{(\Varid{y},\Varid{h}\;(\Varid{x},\Varid{ys}))}
\just\equiv{ Def-x, Cancelamento-+ }
\ensuremath{(\Varid{id}\times\alt{\Varid{h}}{\Varid{k}})\;(\Varid{y},i_1\;(\Varid{x},\Varid{ys}))}
\just\equiv{ Def-g2 }
\ensuremath{(\Varid{id}\times\alt{\Varid{h}}{\Varid{k}})\comp \Varid{g2}\;(\Varid{x},\Conid{Cons}\;(\Varid{y},\Varid{ys}))}
\qed
\end{eqnarray*}

Com isto fica demonstrado que \ensuremath{\Varid{fair\char95 merge}} se pode definir como um
anamorfismo de \ensuremath{\Conid{Streams}}, ou seja, \ensuremath{\Varid{fair\char95 merge}\mathrel{=}\lanabracket\,\Varid{geneFM}\,\ranabracket},
para o gene \ensuremath{\Varid{geneFM}} definido como \ensuremath{\alt{\Varid{g1}}{\Varid{g2}}}.

Para testar a função \ensuremath{\Varid{fair\char95 merge'}}, definimos a seguinte
função auxiliar \ensuremath{\Varid{takeStream}} que permite extrair os primeiros \ensuremath{\Varid{n}} elementos
de uma \ensuremath{\Conid{Stream}} e devolvê-los como uma lista.

\begin{hscode}\SaveRestoreHook
\column{B}{@{}>{\hspre}l<{\hspost}@{}}%
\column{E}{@{}>{\hspre}l<{\hspost}@{}}%
\>[B]{}\Varid{takeStream}\mathbin{::}\Conid{Int}\to \Conid{Stream}\;\Varid{a}\to [\mskip1.5mu \Varid{a}\mskip1.5mu]{}\<[E]%
\\
\>[B]{}\Varid{takeStream}\;\mathrm{0}\;\anonymous \mathrel{=}[\mskip1.5mu \mskip1.5mu]{}\<[E]%
\\
\>[B]{}\Varid{takeStream}\;\Varid{n}\;(\Conid{Cons}\;(\Varid{x},\Varid{xs}))\mathrel{=}\Varid{x}\mathbin{:}\Varid{takeStream}\;(\Varid{n}\mathbin{-}\mathrm{1})\;\Varid{xs}{}\<[E]%
\ColumnHook
\end{hscode}\resethooks

Definimos também as seguintes \ensuremath{\Conid{Streams}} de exemplo:

\begin{hscode}\SaveRestoreHook
\column{B}{@{}>{\hspre}l<{\hspost}@{}}%
\column{E}{@{}>{\hspre}l<{\hspost}@{}}%
\>[B]{}\Varid{s1}\mathbin{::}\Conid{Stream}\;\Conid{Int}{}\<[E]%
\\
\>[B]{}\Varid{s1}\mathrel{=}\Conid{Cons}\;(\mathrm{1},\Conid{Cons}\;(\mathrm{3},\Conid{Cons}\;(\mathrm{5},\Conid{Cons}\;(\mathrm{7},\Conid{Cons}\;(\mathrm{9},\Varid{s1}))))){}\<[E]%
\\
\>[B]{}\Varid{s2}\mathbin{::}\Conid{Stream}\;\Conid{Int}{}\<[E]%
\\
\>[B]{}\Varid{s2}\mathrel{=}\Conid{Cons}\;(\mathrm{2},\Conid{Cons}\;(\mathrm{4},\Conid{Cons}\;(\mathrm{6},\Conid{Cons}\;(\mathrm{8},\Conid{Cons}\;(\mathrm{10},\Varid{s2}))))){}\<[E]%
\ColumnHook
\end{hscode}\resethooks

A aplicação de \ensuremath{\Varid{takeStream}\;\mathrm{10}\;(\Varid{fair\char95 merge'}\;(i_1\;(\Varid{s1},\Varid{s2})))} resulta na lista:
\begin{hscode}\SaveRestoreHook
\column{B}{@{}>{\hspre}l<{\hspost}@{}}%
\column{E}{@{}>{\hspre}l<{\hspost}@{}}%
\>[B]{}[\mskip1.5mu \mathrm{1},\mathrm{2},\mathrm{3},\mathrm{4},\mathrm{5},\mathrm{6},\mathrm{7},\mathrm{8},\mathrm{9},\mathrm{10}\mskip1.5mu]{}\<[E]%
\ColumnHook
\end{hscode}\resethooks
refletindo a intercalação justa dos elementos das duas \ensuremath{\Conid{Streams}}, 
começando pela primeira.

A aplicação de \ensuremath{\Varid{takeStream}\;\mathrm{10}\;(\Varid{fair\char95 merge'}\;(i_2\;(\Varid{s1},\Varid{s2})))} resulta na lista:
\begin{hscode}\SaveRestoreHook
\column{B}{@{}>{\hspre}l<{\hspost}@{}}%
\column{E}{@{}>{\hspre}l<{\hspost}@{}}%
\>[B]{}[\mskip1.5mu \mathrm{2},\mathrm{1},\mathrm{4},\mathrm{3},\mathrm{6},\mathrm{5},\mathrm{8},\mathrm{7},\mathrm{10},\mathrm{9}\mskip1.5mu]{}\<[E]%
\ColumnHook
\end{hscode}\resethooks
refletindo a intercalação justa dos elementos das duas \ensuremath{\Conid{Streams}}, 
começando pela segunda.
\subsection*{Problema 4}

\textbf{Introdução}

De uma forma simples, o objetivo deste problema é desenhar uma função \ensuremath{\Varid{transmitir}} que descreve o comportamento de um aparelho de telegrafia avariado. Esse aparelho tenta transmitir uma mensagem palavra a palavra, mas pode falhar de forma aleatória: cada palavra pode perder-se durante a transmissão e, no final da mensagem, o envio do código "stop" também pode falhar.

Pretende-se que a função \ensuremath{\Varid{transmitir}} modele corretamente estas falhas, produzindo todas as mensagens possíveis, cada uma associada à respetiva probabilidade. Para tal, recorre-se a um catamorfismo probabilístico sobre listas, \ensuremath{\Varid{pcataList}}, cujo resultado é uma distribuição de probabilidades, representada pelo mónade Dist.

O comportamento local do aparelho é descrito por um \ensuremath{\Varid{gene}}, que define as decisões probabilísticas a tomar em cada passo da transmissão. A função \ensuremath{\Varid{transmitir}\mathrel{=}\Varid{pcataList}\;\Varid{gene}} separa, assim, o mecanismo genérico de percorrer a lista, do comportamento específico do aparelho.

Antes de apresentar a solução completa, achamos útil analisar cada componente que será usado para que, de seguida, a definição do gene seja mais intuitiva de deduzir e possa ser aplicada para verificar a solução para o problema proposto.

\textbf{Uso de Dist}

Para este problema, é necessário modelar a transmissão de mensagens, em que cada palavra pode falhar de forma aleatória. Para isso, precisamos de uma forma de representar todas as mensagens possíveis e associar a cada uma a sua probabilidade de ocorrência.

O mónade Dist faz exatamente isso:

\begin{center}
\begin{hscode}\SaveRestoreHook
\column{B}{@{}>{\hspre}l<{\hspost}@{}}%
\column{E}{@{}>{\hspre}l<{\hspost}@{}}%
\>[B]{}\mathbf{newtype}\;\fun{Dist}\;\Varid{a}\mathrel{=}\Conid{D}\;\{\mskip1.5mu \Varid{unD}\mathbin{::}[\mskip1.5mu (\Varid{a},\Conid{ProbRep})\mskip1.5mu]\mskip1.5mu\}{}\<[E]%
\ColumnHook
\end{hscode}\resethooks
\end{center}

\begin{itemize}
     \item Cada \ensuremath{\Varid{a}} é um possível valor (no caso, uma mensagem ou palavra transmitida).
     \item Cada \ensuremath{\Conid{ProbRep}} é a probabilidade desse valor ocorrer (um número real entre 0 e 1, sendo a soma total das probabilidades 1).
\end{itemize}

Dist forma um mónade porque:

\begin{center}
\begin{itemize}
     \item Tem uma operação return: \ensuremath{\Varid{return}\;\Varid{a}\mathrel{=}\Conid{D}\;[\mskip1.5mu (\Varid{a},\mathrm{1})\mskip1.5mu]}, que gera uma distribuição determinística com probabilidade 1.  
     \item Tem uma composição de Kleisli (bind >>=): 
     \begin{center}
     $(f \bullet g)~a = [(y,q*p) \mid (x,p) \leftarrow g~a, (y,q) \leftarrow f~x]$,  
     \end{center}
     onde 
     
     \begin{center}
     $g :: A \to \texttt{Dist B}$ 
     \end{center}

     e 

     \begin{center}
     $f :: B \to \texttt{Dist C}$ 
     \end{center}
    
     são funções que representam computações probabilísticas.  
\end{itemize}
\end{center}

Esta definição permite combinar automaticamente as probabilidades de múltiplas etapas de cálculo.
Por exemplo, ao percorrer uma lista de palavras, o mónade Dist calcula todas as combinações
possíveis de palavras transmitidas e perdidas, multiplicando as probabilidades associadas a cada passo,
sem necessidade de ciclos explícitos ou cálculos manuais.

Desta forma, alterações no comportamento do \ensuremath{\Varid{gene}} propagam-se corretamente por toda a lista,
sendo suficiente ajustar apenas as probabilidades associadas a cada decisão.

%%%%%%%%%%%%%%%%%%%%%%%%%%%%%%%%%%%%%%%%%%%%%%%%%%%%%%%%%%%%%%%%%%%%%%%%%%%%%%
%%%%%%%%%%%%%%%%%%%%%%%%%%%%%%%%%%%%%%%%%%%%%%%%%%%%%%%%%%%%%%%%%%%%%%%%%%%%%%
%%%%%%%%%%%%%%%%%%%%%%%%%%%%%%%%%%%%%%%%%%%%%%%%%%%%%%%%%%%%%%%%%%%%%%%%%%%%%%
%%%%%%%%%%%%%%%%%%%%%%%%%%%%%%%%%%%%%%%%%%%%%%%%%%%%%%%%%%%%%%%%%%%%%%%%%%%%%%
\textbf{Uso de gene}

O \ensuremath{\Varid{gene}} deverá descrever o comportamento local do aparelho de telegrafia. Para cada palavra, define quais são os resultados que podem ocorrer e com que probabilidades. Neste caso, deverá calcular o caso da perda de uma palavra ou o caso da perda da palavra final "stop".

Pode-se afirmar que o tipo do \ensuremath{\Varid{gene}} é:

\begin{center}
\begin{hscode}\SaveRestoreHook
\column{B}{@{}>{\hspre}l<{\hspost}@{}}%
\column{E}{@{}>{\hspre}l<{\hspost}@{}}%
\>[B]{}\Varid{gene}\mathbin{::}()+(\Conid{String},[\mskip1.5mu \Conid{String}\mskip1.5mu])\to \fun{Dist}\;[\mskip1.5mu \Conid{String}\mskip1.5mu]{}\<[E]%
\ColumnHook
\end{hscode}\resethooks
\end{center}

Já que o uso do \ensuremath{\cdot +\cdot } permite separar de forma natural os dois casos distintos que surgem ao percorrer uma lista:

\begin{itemize}
    \item \textbf{Lista vazia:} representada por \ensuremath{i_1\;()}. Este caso corresponde ao fim da lista, permitindo decidir probabilisticamente se a transmissão termina corretamente ou se ocorre a falha de envio de "stop". 
    \item \textbf{Lista não vazia:} representada por \ensuremath{i_2\;(\Varid{x},\Varid{y})}, onde $x$ é a cabeça da lista (palavra atual) e $y$ é o resultado já processado da cauda. Este caso permite combinar a palavra atual com todos os resultados possíveis da cauda, aplicando as regras probabilísticas definidas.
\end{itemize}

Desta forma, o \ensuremath{\Varid{gene}} consegue tratar de forma modular e uniforme tanto o caso base da lista, quanto a transmissão de cada palavra individual, sem que a função de percorrer a lista (\ensuremath{\Varid{pcataList}}) precise de conhecer detalhes do comportamento probabilístico como: as probabilidades definidas ou possíveis decisões probabilísticas complexas (por exemplo, múltiplas opções para cada palavra ou eventos condicionais).

%%%%%%%%%%%%%%%%%%%%%%%%%%%%%%%%%%%%%%%%%%%%%%%%%%%%%%%%%%%%%%%%%%%%%%%%%%%%%%
%%%%%%%%%%%%%%%%%%%%%%%%%%%%%%%%%%%%%%%%%%%%%%%%%%%%%%%%%%%%%%%%%%%%%%%%%%%%%%
%%%%%%%%%%%%%%%%%%%%%%%%%%%%%%%%%%%%%%%%%%%%%%%%%%%%%%%%%%%%%%%%%%%%%%%%%%%%%%
%%%%%%%%%%%%%%%%%%%%%%%%%%%%%%%%%%%%%%%%%%%%%%%%%%%%%%%%%%%%%%%%%%%%%%%%%%%%%%
\textbf{Uso de pcataList}

O catamorfismo \ensuremath{\Varid{pcataList}} percorre a lista de palavras de forma recursiva, aplicando a função \ensuremath{\Varid{gene}} a cada passo. Esta abordagem permite separar a lógica de percorrer a lista, da lógica de transmissão probabilística, tornando o comportamento do aparelho mais fácil de modelar. Sabendo a declaração de \ensuremath{\Varid{pcataList}} (dada no enunciado):

\begin{center}
\begin{hscode}\SaveRestoreHook
\column{B}{@{}>{\hspre}l<{\hspost}@{}}%
\column{E}{@{}>{\hspre}l<{\hspost}@{}}%
\>[B]{}\Varid{pcataList}\mathbin{::}(()+(\Varid{a},\Varid{b})\to \fun{Dist}\;\Varid{b})\to [\mskip1.5mu \Varid{a}\mskip1.5mu]\to \fun{Dist}\;\Varid{b}{}\<[E]%
\ColumnHook
\end{hscode}\resethooks
\end{center}

E que em listas, os catamorfismos devem ter em conta os casos de lista vazia e não vazia, é possível deduzir a definição de \ensuremath{\Varid{pcataList}} como sendo:

\begin{center}
\begin{hscode}\SaveRestoreHook
\column{B}{@{}>{\hspre}l<{\hspost}@{}}%
\column{5}{@{}>{\hspre}l<{\hspost}@{}}%
\column{23}{@{}>{\hspre}l<{\hspost}@{}}%
\column{E}{@{}>{\hspre}l<{\hspost}@{}}%
\>[B]{}\Varid{pcataList}\;\Varid{gene}\;[\mskip1.5mu \mskip1.5mu]{}\<[23]%
\>[23]{}\mathrel{=}\Varid{gene}\;(i_1\;()){}\<[E]%
\\
\>[B]{}\Varid{pcataList}\;\Varid{gene}\;(\Varid{x}\mathbin{:}\Varid{xs})\mathrel{=}\mathbf{do}{}\<[E]%
\\
\>[B]{}\hsindent{5}{}\<[5]%
\>[5]{}\Varid{y}\leftarrow \Varid{pcataList}\;\Varid{gene}\;\Varid{xs}{}\<[E]%
\\
\>[B]{}\hsindent{5}{}\<[5]%
\>[5]{}\Varid{gene}\;(i_2\;(\Varid{x},\Varid{y})){}\<[E]%
\ColumnHook
\end{hscode}\resethooks
\end{center}

Nesta definição:

\begin{itemize}
    \item No caso da lista ser vazia, \ensuremath{\Varid{gene}} recebe \ensuremath{i_1\;()}, permitindo decidir probabilisticamente se o processo termina.
    \item No caso da lista não ser vazia, a cauda da lista é processada primeiro recursivamente, produzindo $y$, que é então combinado com a cabeça $x$ através de \ensuremath{\Varid{gene}\;(i_2\;(\Varid{x},\Varid{y}))}.
\end{itemize}

Para ilustrar o seu funcionamento, considere a mensagem:
\begin{center}
["hi","hi again","bye"]
\end{center}

A aplicação de \ensuremath{\Varid{pcataList}\;\Varid{gene}} a esta lista é definida recursivamente, seguindo a estrutura do catamorfismo sobre listas. Os passos da execução, são:

\begin{itemize}
     \item \textbf{Passo 1: lista completa ["hi","hi again","bye"]}  
     \begin{center}
     \begin{hscode}\SaveRestoreHook
\column{B}{@{}>{\hspre}l<{\hspost}@{}}%
\column{6}{@{}>{\hspre}l<{\hspost}@{}}%
\column{9}{@{}>{\hspre}l<{\hspost}@{}}%
\column{E}{@{}>{\hspre}l<{\hspost}@{}}%
\>[6]{}\Varid{pcataList}\;\Varid{gene}\;[\mskip1.5mu \text{\ttfamily \char34 hi\char34},\text{\ttfamily \char34 hi~again\char34},\text{\ttfamily \char34 bye\char34}\mskip1.5mu]\mathrel{=}\mathbf{do}{}\<[E]%
\\
\>[6]{}\hsindent{3}{}\<[9]%
\>[9]{}\Varid{y}\leftarrow \Varid{pcataList}\;\Varid{gene}\;[\mskip1.5mu \text{\ttfamily \char34 hi~again\char34},\text{\ttfamily \char34 bye\char34}\mskip1.5mu];{}\<[E]%
\\
\>[6]{}\hsindent{3}{}\<[9]%
\>[9]{}\Varid{gene}\;(i_2\;(\text{\ttfamily \char34 hi\char34},\Varid{y})){}\<[E]%
\ColumnHook
\end{hscode}\resethooks
     \end{center}
    O valor $y$ representa uma distribuição de todos os resultados possíveis da cauda ["hi again","bye"].

     \item \textbf{Passo 2: cauda ["hi again","bye"]}  
     \begin{center}
     \begin{hscode}\SaveRestoreHook
\column{B}{@{}>{\hspre}l<{\hspost}@{}}%
\column{6}{@{}>{\hspre}l<{\hspost}@{}}%
\column{9}{@{}>{\hspre}l<{\hspost}@{}}%
\column{E}{@{}>{\hspre}l<{\hspost}@{}}%
\>[6]{}\Varid{pcataList}\;\Varid{gene}\;[\mskip1.5mu \text{\ttfamily \char34 hi~again\char34},\text{\ttfamily \char34 bye\char34}\mskip1.5mu]\mathrel{=}\mathbf{do}{}\<[E]%
\\
\>[6]{}\hsindent{3}{}\<[9]%
\>[9]{}\Varid{y1}\leftarrow \Varid{pcataList}\;\Varid{gene}\;[\mskip1.5mu \text{\ttfamily \char34 bye\char34}\mskip1.5mu];{}\<[E]%
\\
\>[6]{}\hsindent{3}{}\<[9]%
\>[9]{}\Varid{gene}\;(i_2\;(\text{\ttfamily \char34 hi~again\char34},\Varid{y1})){}\<[E]%
\ColumnHook
\end{hscode}\resethooks
     \end{center}
    Cada resultado $y_1$ da chamada à cauda anterior é combinado com "hi again" através do \ensuremath{\Varid{gene}}, propagando todas as combinações possíveis com as probabilidades corretas.

     \item \textbf{Passo 3: cauda ["bye"]}  
     \begin{center}
     \begin{hscode}\SaveRestoreHook
\column{B}{@{}>{\hspre}l<{\hspost}@{}}%
\column{6}{@{}>{\hspre}l<{\hspost}@{}}%
\column{9}{@{}>{\hspre}l<{\hspost}@{}}%
\column{E}{@{}>{\hspre}l<{\hspost}@{}}%
\>[6]{}\Varid{pcataList}\;\Varid{gene}\;[\mskip1.5mu \text{\ttfamily \char34 bye\char34}\mskip1.5mu]\mathrel{=}\mathbf{do}{}\<[E]%
\\
\>[6]{}\hsindent{3}{}\<[9]%
\>[9]{}\Varid{y2}\leftarrow \Varid{pcataList}\;\Varid{gene}\;[\mskip1.5mu \mskip1.5mu];{}\<[E]%
\\
\>[6]{}\hsindent{3}{}\<[9]%
\>[9]{}\Varid{gene}\;(i_2\;(\text{\ttfamily \char34 bye\char34},\Varid{y2})){}\<[E]%
\ColumnHook
\end{hscode}\resethooks
     \end{center}
    Cada resultado $y_2$ do caso base é combinado com "bye" através do \ensuremath{\Varid{gene}}, multiplicando as probabilidades de cada etapa.

     \item \textbf{Passo 4: caso base []}  
     \begin{center}
     \begin{hscode}\SaveRestoreHook
\column{B}{@{}>{\hspre}l<{\hspost}@{}}%
\column{6}{@{}>{\hspre}l<{\hspost}@{}}%
\column{E}{@{}>{\hspre}l<{\hspost}@{}}%
\>[6]{}\Varid{pcataList}\;\Varid{gene}\;[\mskip1.5mu \mskip1.5mu]\mathrel{=}\Varid{gene}\;(i_1\;()){}\<[E]%
\ColumnHook
\end{hscode}\resethooks
     \end{center}
    Produz a distribuição inicial que representa as possíveis terminações da mensagem, de acordo com o comportamento probabilístico do \ensuremath{\Varid{gene}}.
\end{itemize}

A partir deste caso base, o catamorfismo combina sucessivamente cada palavra com todos os resultados possíveis da sua cauda, multiplicando as probabilidades locais. Deste modo, o valor

\begin{center}
\begin{hscode}\SaveRestoreHook
\column{B}{@{}>{\hspre}l<{\hspost}@{}}%
\column{E}{@{}>{\hspre}l<{\hspost}@{}}%
\>[B]{}\Varid{pcataList}\;\Varid{gene}\;[\mskip1.5mu \text{\ttfamily \char34 hi\char34},\text{\ttfamily \char34 hi~again\char34},\text{\ttfamily \char34 bye\char34}\mskip1.5mu]{}\<[E]%
\ColumnHook
\end{hscode}\resethooks
\end{center}

corresponde a uma única distribuição final que contém todas as mensagens possíveis resultantes da transmissão, já com as probabilidades corretamente combinadas.

%%%%%%%%%%%%%%%%%%%%%%%%%%%%%%%%%%%%%%%%%%%%%%%%%%%%%%%%%%%%%%%%%%%%%%%%%%%%%%
%%%%%%%%%%%%%%%%%%%%%%%%%%%%%%%%%%%%%%%%%%%%%%%%%%%%%%%%%%%%%%%%%%%%%%%%%%%%%%
%%%%%%%%%%%%%%%%%%%%%%%%%%%%%%%%%%%%%%%%%%%%%%%%%%%%%%%%%%%%%%%%%%%%%%%%%%%%%%
%%%%%%%%%%%%%%%%%%%%%%%%%%%%%%%%%%%%%%%%%%%%%%%%%%%%%%%%%%%%%%%%%%%%%%%%%%%%%%
\textbf{Definição de gene}

Depois de analisados os três componentes principais desta solução — o mónade \ensuremath{\fun{Dist}}, o comportamento encapsulado pelo \ensuremath{\Varid{gene}} e o catamorfismo \ensuremath{\Varid{pcataList}} —, a definição do \ensuremath{\Varid{gene}} torna-se natural.

\begin{itemize}

\item Para o caso da lista ser vazia (\ensuremath{i_1\;()}), que corresponde ao fim da transmissão, podemos definir uma probabilidade de 90\% de enviar corretamente a palavra final "stop" e de 10\% de falhar.
\item Para o caso da lista não ser vazia (\ensuremath{i_2\;(\Varid{x},\Varid{y})}), cada palavra $x$ tem uma probabilidade de 95\% de ser transmitida, enquanto que os restantes 5\% representam a perda da palavra.

\end{itemize}

Assim, fica-se com:


\begin{center}
\begin{hscode}\SaveRestoreHook
\column{B}{@{}>{\hspre}l<{\hspost}@{}}%
\column{E}{@{}>{\hspre}l<{\hspost}@{}}%
\>[B]{}\Varid{gene}\mathbin{::}()+(\Conid{String},[\mskip1.5mu \Conid{String}\mskip1.5mu])\to \fun{Dist}\;[\mskip1.5mu \Conid{String}\mskip1.5mu]{}\<[E]%
\\
\>[B]{}\Varid{gene}\;(i_1\;())\mathrel{=}\Conid{D}\;[\mskip1.5mu ([\mskip1.5mu \text{\ttfamily \char34 stop\char34}\mskip1.5mu],\mathrm{0.9}),([\mskip1.5mu \mskip1.5mu],\mathrm{0.1})\mskip1.5mu]{}\<[E]%
\\
\>[B]{}\Varid{gene}\;(i_2\;(\Varid{x},\Varid{ys}))\mathrel{=}\Conid{D}\;[\mskip1.5mu (\Varid{x}\mathbin{:}\Varid{ys},\mathrm{0.95}),(\Varid{ys},\mathrm{0.05})\mskip1.5mu]{}\<[E]%
\ColumnHook
\end{hscode}\resethooks
\end{center}

%%%%%%%%%%%%%%%%%%%%%%%%%%%%%%%%%%%%%%%%%%%%%%%%%%%%%%%%%%%%%%%%%%%%%%%%%%%%%%
%%%%%%%%%%%%%%%%%%%%%%%%%%%%%%%%%%%%%%%%%%%%%%%%%%%%%%%%%%%%%%%%%%%%%%%%%%%%%%
%%%%%%%%%%%%%%%%%%%%%%%%%%%%%%%%%%%%%%%%%%%%%%%%%%%%%%%%%%%%%%%%%%%%%%%%%%%%%%
%%%%%%%%%%%%%%%%%%%%%%%%%%%%%%%%%%%%%%%%%%%%%%%%%%%%%%%%%%%%%%%%%%%%%%%%%%%%%%
\textbf{Cálculo das soluções}

Tendo \ensuremath{\Varid{gene}} definido, basta correr no interpretador \ensuremath{\Varid{transmitir}\;[\mskip1.5mu \text{\ttfamily \char34 Vamos\char34},\text{\ttfamily \char34 atacar\char34},\text{\ttfamily \char34 hoje\char34}\mskip1.5mu]} para calcular as probabilidades de todas as possíveis frases transmitidas. Fazendo isso, obtêve-se:

\begin{center}
\begin{tabular}{l r}
Mensagem transmitida & Probabilidade \\ \hline
\texttt{["Vamos","atacar","hoje","stop"]} & $77.2\%$ \\
\texttt{["Vamos","atacar","hoje"]} & $8.6\%$ \\
\texttt{["Vamos","atacar","stop"]} & $4.1\%$ \\
\texttt{["Vamos","hoje","stop"]} & $4.1\%$ \\
\texttt{["atacar","hoje","stop"]} & $4.1\%$ \\
\texttt{["Vamos","atacar"]} & $0.5\%$ \\
\texttt{["Vamos","hoje"]} & $0.5\%$ \\
\texttt{["atacar","hoje"]} & $0.5\%$ \\
\texttt{["Vamos","stop"]} & $0.2\%$ \\
\texttt{["atacar","stop"]} & $0.2\%$ \\
\texttt{["hoje","stop"]} & $0.2\%$ \\
\texttt{["atacar"]} & $0.0\%$ \\
\texttt{["hoje"]} & $0.0\%$ \\
\texttt{["Vamos"]} & $0.0\%$ \\
\texttt{["stop"]} & $0.0\%$ \\
\texttt{[]} & $0.0\%$ \\
\end{tabular}
\end{center}

Assim, observando os resultados obtidos, é possível responder às questões propostas:

\begin{itemize}
     \item Qual a probabilidade da palavra "atacar" da mensagem se perder? 
     \textbf {É de 4.1\%.}
     \item Qual a probabilidade de seguirem todas as palavras, mas faltar o "stop"?
     \textbf {É de 8.6\%.}
     \item Qual a probabilidade da transmissão ser perfeita?
     \textbf {É de 77.2\%.}
\end{itemize}































%----------------- Índice remissivo (exige makeindex) -------------------------%

\printindex

%----------------- Bibliografia (exige bibtex) --------------------------------%

\bibliographystyle{plain}
\bibliography{cp2526t}

%----------------- Fim do documento -------------------------------------------%
\end{document}
